% file: Hyperloop_stuff.tex
%

\documentclass[10pt]{amsart}
\pdfoutput=1
\usepackage{mathtools,amssymb,caption}

\usepackage{graphicx}
\usepackage{hyperref}
\usepackage[utf8]{inputenc}
\usepackage{listings}
\usepackage[table]{xcolor}
\usepackage{pdfpages}
\usepackage{tikz}
\usetikzlibrary{matrix,arrows,backgrounds, shapes}

\usepackage{breqn} % for dmath

%\usepackage{cancel} % for Feynman slash notation

\hypersetup{colorlinks=true,citecolor=[rgb]{0,0.4,0}}

%\oddsidemargin=15pt
%\evensidemargin=5pt
%\hoffset-45pt
%\voffset-55pt
%\topmargin=-4pt
%\headsep=5pt
%\textwidth=1120pt
%\textheight=595pt
%\paperwidth=1200pt
%\paperheight=700pt
%\footskip=40pt

\newtheorem{theorem}{Theorem}
\newtheorem{corollary}{Corollary}
%\newtheorem*{main}{Main Theorem}
\newtheorem{lemma}{Lemma}
\newtheorem{proposition}{Proposition}

\newtheorem{definition}{Definition}
\newtheorem{remark}{Remark}

\newenvironment{claim}[1]{\par\noindent\underline{Claim:}\space#1}{}
\newenvironment{claimproof}[1]{\par\noindent\underline{Proof:}\space#1}{\hfill $\blacksquare$}

%This defines a new command \questionhead which takes one argument and
%prints out Question #. with some space.
\newcommand{\questionhead}[1]
  {\bigskip\bigskip
   \noindent{\small\bf Question #1.}
   \bigskip}

\newcommand{\problemhead}[1]
  {
   \noindent{\small\bf Problem #1.}
   }

\newcommand{\exercisehead}[1]
  { \smallskip
   \noindent{\small\bf Exercise #1.}
  }

\newcommand{\solutionhead}[1]
  {
   \noindent{\small\bf Solution #1.}
   }


  \title[On the Hyperloop]{On the Hyperloop}

\author{Ernest Yeung \href{mailto:ernestyalumni@gmail.com}{ernestyalumni@gmail.com}}
\date{17 Avr 2021}
\keywords{Hyperloop, convex optimization}

\begin{document}

\definecolor{darkgreen}{rgb}{0,0.4,0}
\lstset{language=Python,
 frame=bottomline,
 basicstyle=\scriptsize,
 identifierstyle=\color{blue},
 keywordstyle=\bfseries,
 commentstyle=\color{darkgreen},
 stringstyle=\color{red},
 }
%\lstlistoflistings

\maketitle

\tableofcontents

\begin{abstract}
All things about the Hyperloop.

This document is a dump of notes, thoughts, interesting references and links to the Hyperloop.
\end{abstract}

\section{Introduction}


\part{Qualitative Summary}




\begin{thebibliography}{9}



\bibitem{kirschen2021hyperloop}
Philippe Kirschen and Edward Burnell. \textbf{Hyperloop System Optimization}. 
2104.03907. 2021. \url{https://arxiv.org/abs/2104.03907}

\end{thebibliography}

\end{document}
