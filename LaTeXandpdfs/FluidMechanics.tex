% file: FluidMechanics.tex
% Fluid Mechanics
% 
% Typeset with LaTeX format
% cf. Math Into Latex Third Edition pp. 290
% This file has my modifications
%
% github        : ernestyalumni
% gmail         : ernestyalumni 
% linkedin      : ernestyalumni 
% twitter       : ernestyalumni 
% wordpress.com : ernestyalumni
% 
% This code is open-source, governed by the Creative Common license.  Use of this code is governed by the Caltech Honor Code: ``No member of the Caltech community shall take unfair advantage of any other member of the Caltech community.'' 
% 


\documentclass[twoside,landscape,10pt]{amsart}

%\setcounter{tocdepth}{1} % to get subsubsections in toc 
% cf. http://www.latex-community.org/forum/viewtopic.php?f=47&p=44760


\usepackage{amsmath,amssymb,latexsym}
\usepackage{graphics}
\usepackage{tikz}
\usepackage{hyperref}
\usepackage{listings}
\hypersetup{colorlinks=true, urlcolor=blue}

\usetikzlibrary{matrix,arrows}

%\usepackage[parfill]{parskip}

\usepackage{multicol}

%\usepackage{fontspec}
%\setmainfont{Times New Roman} % this sets the real Times New Roman via modern TeX engines

\usepackage{mathptmx}

%\hoffset-

\oddsidemargin=15pt
\evensidemargin=5pt
\hoffset-40pt
\voffset-55pt
\topmargin=-4pt
\headsep=5pt
\textwidth=1120pt
\textheight=620pt
\paperwidth=1200pt
\paperheight=700pt

\marginparwidth=12pt

\parindent0.0em

%\linespread{1.2}

%plain makes sure that we have page numbers
\pagestyle{plain}

\theoremstyle{plain}
\newtheorem{theorem}{Theorem}
\newtheorem{corollary}{Corollary}
%\newtheorem*{main}{Main Theorem}
\newtheorem{lemma}{Lemma}
\newtheorem{proposition}{Proposition}

\theoremstyle{definition}
\newtheorem{definition}{Definition}

\theoremstyle{remark}
\newtheorem*{notation}{Notation}

\theoremstyle{remark}
\newtheorem{remark}{Remark}

%\numberwithin{equation}{section}

%This defines a new command \questionhead which takes one argument and
%prints out Question #. with some space.
\newcommand{\questionhead}[1]
  {\bigskip\bigskip
   \noindent{\small\bf Question #1.}
   \bigskip}

\newcommand{\problemhead}[1]
  {
   \noindent{\small\bf Problem #1.}
   }

\newcommand{\exercisehead}[1]
  { \smallskip
   \noindent{\small\bf Exercise #1.}
  }

\newcommand{\solutionhead}[1]
  {
   \noindent{\small\bf Solution #1.}
   }



%-----------------------------------
\begin{document}
%-----------------------------------
\title[FluidMechanics]{Fluid Mechanics}
\author{Ernest Yeung}
\address{}
\email{ernestyalumni@gmail.com}
\urladdr{http://ernestyalumni.wordpress.com}
\thanks{linkedin : ernestyalumni }

\keywords{Fluid Mechanics}
\subjclass[Fluid Mechanics]{Fluid Mechanics}
\date{17 juillet 2015}
\begin{abstract}
Everything about Fluid Mechanics
\end{abstract}

\definecolor{darkgreen}{rgb}{0,0.4,0}
\lstset{language=Python,
 frame=bottomline,
 basicstyle=\scriptsize,
 identifierstyle=\color{blue},
 keywordstyle=\bfseries,
 commentstyle=\color{darkgreen},
 stringstyle=\color{red},
 }
%\lstlistoflistings

\maketitle



\tableofcontents


\begin{multicols*}{2}

\part{div, grad, curl}

\section{gradient}

Jost (2011) \cite{JJost2011}, in Chapter 3 The Laplace Operator and Harmonic Differential Forms, Section 3.1 The Laplace Operator on Functions, Eqns. (3.1.16)-(3.1.17) on pp. 92, has a definition for gradient on a Riemannian manifold $(M,g)$, along with Calin and Chang (2005) \cite{OCalinDChang2005}, in Chapter 2 Laplace Operators on Riemannian Manifolds, Section 2.1 Gradient vector field; Divergence and Laplacian, Definition 2.1 on pp. 18:
\begin{definition}
  Let $(M,g)$ Riemannian manifold, $f\in C^{\infty}(M)$.  
\begin{equation}
  \begin{aligned}
    &    \text{grad}:C^{\infty}(M) \to \mathfrak{X}(M) \text{ s.t. } \\  
    &  g( \text{grad}f,X) = df(X) \quad \, \forall \, f \in C^{\infty}(M), \, \forall \, X \in \mathfrak{X}(M)
\end{aligned}
\end{equation}
Then locally,
\begin{equation}
g_{ij} (\text{grad}f)^i X^j = \frac{ \partial f}{ \partial x^j} X^j \Longrightarrow (\text{grad}f)^i = g^{ij} \frac{ \partial f}{ \partial x^j}
\end{equation}

\end{definition}

\section{divergence}

Jost (2011) \cite{JJost2011}, in Chapter 3 The Laplace Operator and Harmonic Differential Forms, Section 3.1 The Laplace Operator on Functions, Eq. (3.1.20) on pp. 92, has a definition for divergence on a Riemannian manifold $(M,g)$:
\begin{definition}[divergence]
  $\forall \, $ Riemannian manifold $(M,g)$, 
\begin{equation}
  \begin{aligned}
    & \text{div}: \mathfrak{X}(M) \to C^{\infty}(M) \\ 
    & \text{div}Z = \frac{1}{\sqrt{g}} \frac{ \partial }{ \partial x^j} (\sqrt{g} Z^j ) = *d*Z^{\flat}
\end{aligned}
\end{equation}
\end{definition}
Indeed, since, locally,
\[
\begin{gathered}
  Z^i \frac{ \partial }{ \partial x^i} \overset{\flat}{\mapsto } g_{ij} Z^j dx^i \overset{*}{\mapsto } \frac{ \sqrt{g}}{ (d-1)!} g^{ij} \epsilon_{jj_2 \dots j_d} Z_i dx^{j_2} \wedge \dots \wedge dx^{j_d} \overset{d}{\mapsto } \frac{1}{\sqrt{g}} \frac{ \partial ( \sqrt{g} Z^j ) }{ \partial x^j} \text{vol}^d \overset{*}{\mapsto } \frac{1}{\sqrt{g}} \frac{ \partial (\sqrt{g} Z^j)}{ \partial x^j}
\end{gathered}
\]

Calin and Chang (2005) \cite{OCalinDChang2005}, in Chapter 2 Laplace Operators on Riemannian Manifolds, Section 2.1 Gradient vector field; Divergence and Laplacian, Definition 2.5 on pp. 18, had a definition for divergence, but in terms of an orthonormal frame for the local coordinates, to make their definition of divergence work: here, I'll consider the local \emph{coordinate} frame:
\begin{definition}[divergence]
\begin{equation}
  \text{div}X := dx^j (\nabla_j X )
\end{equation}
and so 
\[
  \text{div}X := dx^j(\nabla_j X ) = dx^j \left( \left( \frac{ \partial X^k}{ \partial x^j} + \Gamma^k_{ \; \; ij} X^i \right) \frac{ \partial }{ \partial x^k}  \right) = \left( \frac{ \partial X^j}{ \partial x^j} + \Gamma^j_{ \; \; ij} X^i \right) 
\]
\end{definition}
%\sum_j g\left( \left( \frac{ \partial X^k}{ \partial x^j} + \Gamma^k_{ \; \; ij} X^i \right) \frac{ \partial }{ \partial x^k}, \frac{ \partial }{ \partial x^j} \right) = \sum_j g_{kj} \left( \frac{ \partial X^k}{ \partial x^j} + \Gamma^k_{ \; \; ij} X^i \right)

This is Lemma 2.6 of Calin and Chang (2005) \cite{OCalinDChang2005}, in Chapter 2 Laplace Operators on Riemannian Manifolds, Section 2.1 Gradient vector field; Divergence and Laplacian on pp. 19, but proven using the local coordinate frame, since I can easily use the unique existence of a Levi-Civita connection, on any Riemannian manifold.    
\begin{lemma}
  locally,
\begin{equation}
  \text{div}X = \sum_j g(\nabla_j X, \frac{ \partial }{ \partial x^j} ) = \frac{1}{\sqrt{g}} \frac{ \partial }{ \partial x^j} (\sqrt{g} X^j )
\end{equation}
\end{lemma}

\begin{proof}
  By theorem, in full generality, $\forall \, $ Riemannian manifold $(M,g)$, $\exists \, !$ Levi-Civita connection locally expressed as 
\[
\Gamma^c_{ \; \; ba } = \frac{1}{2} g^{cm} \left( \frac{ \partial g_{bm}}{ \partial x^a } + \frac{ \partial g_{ma}}{ \partial x^b} - \frac{ \partial g_{ab}}{ \partial x^m} \right)
\]
following the convention by Dr. Schuller \footnote{Frederic Schuller, The WE-Heraeus International Winter School on Gravity and Light, ``Lecture 10: Metric Manifolds (International Winter School on Gravity and Light 2015)'', \url{https://youtu.be/ONCZNwKswn4}}

Rewriting the indices:
\[
\Gamma^k_{ \; \; ij } = \frac{1}{2} g^{kl} \left( \frac{ \partial g_{il}}{ \partial x^j } + \frac{ \partial g_{lj}}{ \partial x^i} - \frac{ \partial g_{ij}}{ \partial x^l} \right)
\]
Then
%\[
%\begin{gathered}
%g_{kj}\Gamma^k_{ \; \; ij }X^i = g_{kj} \frac{1}{2} g^{kl} \left( \frac{ \partial g_{il}}{ \partial x^j } + \frac{ \partial g_{lj}}{ \partial x^i} - \frac{ \partial g_{ij}}{ \partial x^l} \right) X^i = g_{kj} \frac{1}{2} g^{kl} \left( \frac{ \partial g_{il}}{ \partial x^j } + \frac{ \partial g_{lj}}{ \partial x^i} - \frac{ \partial g_{ij}}{ \partial x^l} \right) X^i = \frac{1}{2} \delta^l_j \left( \frac{ \partial g_{il}}{ \partial x^j } + \frac{ \partial g_{lj}}{ \partial x^i} - \frac{ \partial g_{ij}}{ \partial x^l} \right) X^i = \frac{1}{2} \frac{ \partial g_{jj}}{ \partial x^i} X^i 
%\end{gathered}
%\]

\[
\begin{gathered}
\Gamma^j_{ \; \; ij }X^i =  \frac{1}{2} g^{jl} \left( \frac{ \partial g_{il}}{ \partial x^j } + \frac{ \partial g_{lj}}{ \partial x^i} - \frac{ \partial g_{ij}}{ \partial x^l} \right) X^i =  \frac{1}{2} g^{jl} \left( \frac{ \partial g_{lj}}{ \partial x^i }X^i + \frac{1}{2} g^{jl}\frac{ \partial g_{il}}{ \partial x^j}X^i - \frac{1}{2}g^{jl} \frac{ \partial g_{ij}}{ \partial x^l} \right) X^i = \frac{1}{2} g^{jl} \frac{ \partial g_{lj} }{ \partial x^i} X^i
\end{gathered}
\]


Let $g\equiv \text{det}g = g(g_{11}, g_{12} \dots g_{ij} \dots g_{nn})$ i.e. $g\in C^{\infty}(\mathbb{R}^{n^2})$.  Now by product rule (1-dimensional calculus)
\[
\frac{ \partial g}{ \partial x^k} = \frac{ \partial g}{ \partial g_{ij} } \frac{ \partial g_{ij}}{ \partial x^k}
\]
$\frac{ \partial g}{ \partial g_{ij}}$ is the minor of $g_{ij}$ (recall your undergraduate linear algebra of computing determinants explicitly).  

Then
\[
g^{ij} = \frac{1}{g} \frac{ \partial g}{ \partial g_{ij}}
\]
where $g^{ij}$ is the inverse matrix of $g_{ij}$ (again, recall your undergraduate linear algebra for matrices).  

Then
\[
\frac{ \partial g}{ \partial x^k} = gg^{ij} \frac{ \partial g_{ij}}{ \partial x^k}
\]
and so 
\[
\Gamma^j_{ \; \; ij} X^i = \frac{1}{2} \frac{1}{g} \frac{ \partial g}{ \partial x^i} X^i
\]

Now 
\[
\frac{1}{\sqrt{g}} \frac{ \partial }{ \partial x^k} (\sqrt{g} X^k)  = \frac{ \partial X^k}{ \partial x^k} + \frac{1}{\sqrt{g}} \frac{1}{2} \frac{1}{\sqrt{g}} \frac{ \partial g}{ \partial x^k} X^k = \frac{ \partial X^k}{ \partial x^k} + \frac{1}{2g} \frac{ \partial g}{ \partial x^k} X^k = \frac{ \partial X^k}{ \partial x^k} + \Gamma^j_{ \; \; kj } X^k
\]
and since
\[
dx^j( \nabla_jX) = dx^j \left( \left( \frac{ \partial X^k}{ \partial x^j} + \Gamma^k_{ \; \; ij} X^i \right) \frac{ \partial }{ \partial x^k} \right) = \frac{ \partial X^j}{ \partial x^j} + \Gamma^j_{ \; \; ij} X^i 
\]
then
\[
dx^j(\nabla_jX) = \frac{1}{\sqrt{g}} \frac{ \partial }{ \partial x^k} (\sqrt{g} X^k)
\]

\end{proof}

\begin{remark}
It's remarkable that these definitions for the divergence $\text{div}$ are equivalent:
\begin{equation}
  \boxed{ 
    \text{div}X = dx^j\left( \nabla_j X \right) = \frac{ \partial X^j}{ \partial x^j} + \Gamma^j_{ \; \; ij} X^i = \frac{1}{\sqrt{g}} \frac{ \partial }{ \partial x^k} (\sqrt{g} X^k) = (*d*)X^{\flat}
    }
\end{equation}
\end{remark}

\section{curl}

I seek a global and invariant definition or description of $\emph{curl}$ on a Riemannian manifold $(M,g)$.  

Consider a metric connection $D$ on vector bundle $E\to M$, with bundle metric $\langle \cdot , \cdot \rangle$, s.t., by definition of a metric connection, $d\langle s,t\rangle = \langle Ds,t \rangle + \langle s,Dt\rangle \quad \, \forall \, s,t \in \Gamma(E)$.  Then $D=d+A$, and $A$ either is skew-symmetric or anti-Hermitian, for a field $\mathbb{K} = \mathbb{R}$ or $\mathbb{C}$, respectively (cf. Jost (2011) \cite{JJost2011}).  

  

\begin{lemma}
\end{lemma}

\begin{proof}
The definition of a metric connection directly leads to this relation:
\[
V\langle X,U \rangle = \langle D_VX,U\rangle + \langle X,D_VU \rangle
\]
Reversing $U,V$:
\[
U\langle X,V \rangle = \langle D_UX,V\rangle + \langle X,D_UV \rangle
\]
Then 
\[
V\langle X,U \rangle - U\langle X,V \rangle = \langle D_VX,U\rangle - \langle D_UX,V\rangle  + \langle X,D_VU \rangle - \langle X,D_UV \rangle
\]


\end{proof}


\part{On Landau and Lifshitz's \textbf{Fluid Mechanics}}\cite{LLandauELifshitz1987}

\section{Ideal Fluids} Chapter 1 of Landau and Lifshitz (1987) \cite{LLandauELifshitz1987}.  

Use the \textbf{Eulerian} or \text{spatial velocity}, the velocity at time $t$ of the particle currently in position $\mathbf{x}$ \cite{DHolmTSchmahCStoica2009} which is a time-dependent velocity field:

Let $N$ be the spatial manifold, with spacetime manifold $M = \mathbb{R} \times N$, $\text{dim}N=n$  

\begin{tikzpicture}
  \matrix (m) [matrix of math nodes, row sep=3.8em, column sep=4.8em, minimum width=2.2em]
  {
TN         & u(t,x)         \\
J\times N  & (t,x)         \\
};
  \path[->]
  (m-2-1) edge node [auto] {$u$} (m-1-1)
;
  \path[|->]
  (m-2-2) edge node [auto]  {$u$} (m-1-2)
  ;
\end{tikzpicture}

cf. Section 11.2 ``Geometric setting of ideal continuum motion'' of Holm, Schmah, Stoics \cite{DHolmTSchmahCStoica2009}
\begin{definition}
  Let spacetime manifold $M$ admit a time-foliation $M = \mathbb{R} \times N$, where $N$ represents spatial points. 
Let domain $\mathcal{D} \subseteq N$ represent positions of material particles of system in its \textbf{reference configuration.} Coordinate function $(a^i)$ on $\mathcal{D}$ represent \textbf{particle labels}.  

\begin{itemize}
  \item \textbf{configuration} $:= $ diffeomorphism $g: \mathcal{D} \to \mathcal{D}$, $g\in \text{Diff}(\mathcal{D})$ space of diffeomorphisms from $\mathcal{D}$ to $\mathcal{D}$ 
\item \textbf{fluid motion} $g_t \equiv g(t) \in \text{Diff}(\mathcal{D})$
\end{itemize}  
\end{definition}

\begin{definition}
  \begin{equation}
    \begin{aligned}
      & x: \mathcal{D} \times \mathbb{R} \to \mathcal{D} \\ 
      & x(a,t) := g_t(a) = g(t)\cdot a \in \mathcal{D}      
\end{aligned}
\end{equation} describes path in $\mathcal{D}$ by a particle labeled $a \in \mathcal{D}$
\end{definition}

\begin{definition}
\textbf{Lagrangian or material velocity} - keep particle labels $a$ fixed.  
\begin{equation}
  U(a,t) := \frac{ \partial }{ \partial t} g_t \cdot a = \frac{ \partial }{ \partial t} x(a,t)
\end{equation}
$U(a,t)$ is velocity of particle with label $a$ at time $t$.  


\textbf{Eulerian or spatial velocity } $u$ \\
\phantom{\quad \, } if $x = x(a,t) = g_t(a)$ 
\begin{equation}
u(x,t) := U(a,t) = U(g^{-1}_t(x),t)
\end{equation}
$u(x,t)$ is velocity at time $t$ of particle currently in position $x$
\end{definition}

Now $u \Longrightarrow \begin{aligned} & \quad \\
  & u_t \in \mathfrak{X}(\mathcal{D}) \\
  & u_t(x) := u(x,t) \end{aligned}$ 

$U_t(a) := U(a,t)$ though this isn't really a vector field.  
\[
U_t = u_t \circ g_t
\]


\begin{tikzpicture}
  \matrix (m) [matrix of math nodes, row sep=3.8em, column sep=4.8em, minimum width=2.2em]
  {
& \mathfrak{X}(\mathcal{D}) \\
    \mathcal{D} \subseteq N & \mathcal{D} \subseteq N \\
};
  \path[->]
  (m-2-1) edge node [above] {$U_t$} (m-1-2)
          edge node [above] {$g(t) = g_t$} (m-2-2)
  (m-2-2) edge node [auto]  {$$} (m-1-2)
          edge [bend left=30] node [below] {$g_t^{-1} = g_{-t}$} (m-2-1)
  ;
\end{tikzpicture}  \quad \quad \quad \begin{tikzpicture}
  \matrix (m) [matrix of math nodes, row sep=3.8em, column sep=4.8em, minimum width=2.2em]
  {
& u_t(x):= u(x,t) = U(a,t) =: U_t(a)    \\
    a  & x(a,t)=x   \\
};
  \path[|->]
  (m-2-1) edge node [above] {$U_t $} (m-1-2)
          edge node [above] {$g(t) = g_t$} (m-2-2)
  (m-2-2) edge node [auto]  {$$} (m-1-2)
          edge [bend left=30] node [below] {$g_t^{-1} = g_{-t}$} (m-2-1)
  ;
\end{tikzpicture}  


\begin{definition}
  Given path $g(t) \in \text{Diff}(\mathcal{D})$, \\
  Lagrangian velocity field $\begin{aligned} & \quad \\ 
    & U_t: \mathcal{D} \to \mathfrak{X}(\mathcal{D}) \\
    & U_t \equiv \dot{g}(t) \equiv \frac{ \partial g(t) }{ \partial t} \end{aligned}$ 
\[
\dot{g}(t) \cdot a := \dot{g}(t)(a) = U_t(a) = \frac{ \partial g_t}{ \partial t} \cdot a
\]

\end{definition}

\begin{tikzpicture}
  \matrix (m) [matrix of math nodes, row sep=3.8em, column sep=4.8em, minimum width=2.2em]
  {
\begin{aligned} 
  & g: \mathbb{R} \to \text{Diff}(\mathcal{D}) \\ 
  & g(t) \in \text{Diff}(\mathcal{D})
\end{aligned}  & \text{Diff}(\mathcal{D}) \\
    \dot{g}(t) \equiv \frac{ \partial g(t)}{ \partial t} \equiv U_t  &  T_g\text{Diff}(\mathcal{D})    \\
};
  \path[|->]
  (m-1-1) edge node [left] {$\frac{\partial}{\partial t}$} (m-2-1)
  (m-1-2) edge node [left]  {$\frac{\partial}{\partial t}$} (m-2-2)
  ;
\end{tikzpicture}  

\begin{tikzpicture}
  \matrix (m) [matrix of math nodes, row sep=3.8em, column sep=7.8em, minimum width=2.2em]
  {
a \in \mathcal{D} & U_t(a) \in T_{g_t(a)}\mathcal{D} \\
& g(t)\cdot a = x(a,t) \in \mathcal{D} \\ 
};
  \path[|->]
  (m-1-1) edge node [above] {$\dot{g}(t) \equiv \frac{ \partial g(t)}{ \partial t} \equiv U_t$} (m-1-2)
          edge node [left] {$g(t) = g_t$} (m-2-2)
  (m-2-2) edge node [auto]  {$u_t$} (m-1-2)
         ;
\end{tikzpicture}  

Thus
\[
U_t = u_t \circ g_t
\]
\begin{equation}
  \begin{aligned}
    T_g\text{Diff}(\mathcal{D}) & = \lbrace u \circ g | u \in \mathfrak{X}(\mathcal{D}) \rbrace = \\
    & = \lbrace \text{ smooth } U : \mathcal{D} \to T \mathcal{D} | U(a) \in T_{g(a)}\mathcal{D} \quad \, \forall \, a \in \mathcal{D} \rbrace
\end{aligned}
\end{equation}

\begin{theorem}[(Tangent lift of right translation)]
  Let $\varphi \in \text{Diff}(\mathcal{D})$ \\
Let $R_{\varphi}$ be right translation map \\
\phantom{Let } $ R_{\varphi}: \text{Diff}(\mathcal{D}) \to \text{Diff}(\mathcal{D}) $\\
\phantom{Let } $ R_{\varphi}: g \mapsto g\circ \varphi$ 

tangent life of $R_{\varphi}$ is map $TR_{\varphi} : T\text{Diff}(\mathcal{D}) \to T\text{Diff}(\mathcal{D})$
\begin{equation}
  TR_{\varphi}(U) = TR_{\varphi} \left( \left. \frac{d}{dt} \right|_{t_0} g_t \right) := \left. \frac{d}{dt} \right|_{t_0} (g_t \circ \varphi) = U\circ \varphi
\end{equation}
since $\forall \, a \in \mathcal{D}$

\begin{equation}
  \left. \frac{d}{dt} \right|_{t_0} (g_t \circ \varphi)(a) = \left. \frac{d}{dt} \right|_{t_0} (g_t \circ \varphi(a)) = \left( \left. \frac{d}{dt} \right|_{t_0} g_t \right) \cdot \varphi(a) = U\circ \varphi(a)
\end{equation}

$U\varphi \equiv TR_{\varphi}(U)$

\textbf{Eulerian velocity} corresponding to flow $g(t)$ is $u_t = \dot{g}(t) g^{-1}(t)$

\end{theorem}

\begin{tikzpicture}
  \matrix (m) [matrix of math nodes, row sep=3.8em, column sep=7.8em, minimum width=2.2em]
  {
T\text{Diff}(\mathcal{D} ) & T\text{Diff}(\mathcal{D} ) \\ 
 \text{Diff}(\mathcal{D}) &   \text{Diff}(\mathcal{D}) \\
};
  \path[->]
  (m-1-1) edge node [above] {$TR_{\varphi} $} (m-1-2)
  (m-2-1) edge node [auto]  {$$} (m-1-1)
          edge node [auto]  {$R_{\varphi}$} (m-2-2)
  (m-2-2) edge node [auto] {$$} (m-1-2)
         ;
\end{tikzpicture}  
 \quad \quad \,  \begin{tikzpicture}
  \matrix (m) [matrix of math nodes, row sep=3.8em, column sep=7.8em, minimum width=2.2em]
  {
\left. \frac{d}{dt} \right|_{t_0} g_t = U   & U\circ \varphi = \left. \frac{d}{dt} \right|_{t_0} (g_t \circ \varphi) \\
g & g\circ \varphi   \\
};
  \path[|->]
  (m-1-1) edge node [above] {$TR_{\varphi} $} (m-1-2)
  (m-2-1) edge node [auto]  {$$} (m-1-1)
          edge node [auto]  {$R_{\varphi}$} (m-2-2)
  (m-2-2) edge node [auto] {$$} (m-1-2)
         ;
\end{tikzpicture}  

cf. Section 5.2 ``Abstract Lie groups and Lie algebras'' of Holm, Schmah, Stoics \cite{DHolmTSchmahCStoica2009}




The mass of fluid in some volume $V_0 \subset N$ is $\int_{V^0} \rho \text{vol}^n$, where $\rho$ is fluid density, $\rho \in C^{\infty}(N)$.  

The total mass of fluid flowing out of volume $V_0$ is 
\[
\begin{gathered}
  \frac{d}{dt} \int_{V_0} \rho \text{vol}^n = \int_{V_0} \mathcal{L}_{\frac{\partial}{\partial t} + \textbf{u}} (\rho \text{vol}^n) = \int_{V_0} \frac{ \partial }{\partial t} \rho \text{vol}^n + \int_{V_0} \mathcal{L}_u \rho \text{vol}^n   \\
 \int_{V_0} \mathcal{L}_u \rho \text{vol}^n = \int_{V_0} di_{\mathbf{u}} \rho \text{vol}^n  + i_{\mathbf{u}} d\rho \text{vol}^n = \int_{V_0} di_{\mathbf{u}} \rho \text{vol}^n + 0 = \int_{V_0} di_{\mathbf{u}} \rho \text{vol}^n = \int_{\partial V_0} i_{\mathbf{u}} \rho \text{vol}^n
\end{gathered}
\]  
Now
\[
\begin{gathered}
  i_u \text{vol}^n = i_u\frac{\sqrt{g}}{n!} \epsilon_{i_1 \dots i_n} dx^{i_1} \wedge \dots \wedge dx^{i_n} \\ 
  i_u dx^{i_1} \wedge \dots \wedge dx^{i_n} = u^{i_1} dx^{i_2} \wedge \dots \wedge dx^{i_n} - dx^{i_1} \wedge u^{i_2} dx^{i_3} \wedge \dots \wedge dx^{i_n} +  \dots + (-1)^{n+1} dx^{i_1} \wedge \dots \wedge dx^{i_{n-1}} u^{i_n} = \epsilon^{i_1 \dots i_n}_{j_1 \dots j_n } u^{j_1 } dx^{j_2} \wedge \dots \wedge dx^{j_n}  \\
\Longrightarrow i_u\text{vol}^n = \frac{ \sqrt{g}}{ (n-1)!} \epsilon_{j_1 \dots j_n} u^{j_1} dx^{j_2} \wedge \dots \wedge dx^{j_n}
\end{gathered} 
\]
If $\sqrt{g} = 1$, $n=2$, 
\[
i_u \text{vol}^2 = (u^1 dx^2 - u^2 dx^1) = u\cdot n_1 dx^2 + u\cdot n_2 dx^1 = u\cdot n dS
\]
with $n_1 =e_1$ and $n_2=-e_2$.  

Now 
\[
\begin{gathered}
  di_u \rho \text{vol}^n = \\
  = \frac{ \partial ( \sqrt{g} \rho u^{j_1} ) }{ \partial x^k} \frac{ \epsilon_{j_1 \dots j_n} }{ (n-1)! } dx^k \wedge dx^{j_2} \wedge \dots \wedge dx^{j_n} = \frac{ \partial (\sqrt{ g} \rho u^k) }{ \partial x^k} \frac{ \epsilon_{j_1 \dots j_n }}{ n!} dx^{j_1} \wedge \dots \wedge dx^{j_n} = \frac{1}{\sqrt{g}} \frac{ \partial (\sqrt{g} \rho u^k)}{ \partial x^k} \text{vol}^n = \\
  = \frac{ \partial (\rho u^k)}{ \partial x^k} \text{vol}^n + \rho u^k \frac{ \partial \ln{ \sqrt{g}}}{ \partial x^k} \text{vol}^n = \text{div}(\rho u) \text{vol}^n + \rho u^k \frac{ \partial \ln{ \sqrt{g}}}{ \partial x^k} \text{vol}^n
\end{gathered}
\]
Now if $\sqrt{g}=1$, then 
\[
\begin{gathered}
  \frac{d}{dt} \int_{V_0} \rho \text{vol}^n = \int_{V_0} \frac{ \partial \rho }{ \partial t} \text{vol}^n + \int_{V_0} di_u \rho \text{vol}^n = \int_{V_0} \frac{ \partial \rho }{ \partial t} \text{vol}^n + \int_{V_0} \text{div}(\rho u) \text{vol}^n \Longrightarrow \frac{ \partial \rho}{\partial t} + \text{div}(\rho u) = 0
\end{gathered}
\]
which is the so-called mass continuity equation.  $ j = \rho u$ is the mass flux density.  

cf. Sec.1.2. Euler's equation \cite{LLandauELifshitz1987}, 

The Cauchy-stress tensor $T$ is a symmetric $(0,2)$ tensor, so that $T \in \Gamma(T^*M \otimes T^*M)$.  

Let the normal field $n$ be normal to the surface of $V_0$.  

Let's do this:

\begin{tikzpicture}
  \matrix (m) [matrix of math nodes, row sep=3.8em, column sep=4.8em, minimum width=2.2em]
  {
T & T(n,-) \in T^*M = \Omega^1(M) & *T(n,-) \in \Omega^{m-1}(M) \\ 
};
  \path[->]
  (m-1-1) edge node [above] {$(n,-)$} (m-1-2)
  (m-1-2) edge node [auto]  {$*$} (m-1-3)
  ;
\end{tikzpicture}  

So applying the Hodge operator, 
\[
pg_{ij} n^i dx^j \overset{*}{\mapsto} \frac{ \sqrt{g}}{(n-1)!} pg_{ij} n^i \epsilon^j_{ \,\, j_2 \dots j_n} dx^{j_2} \wedge \dots \wedge dx^{j_n} = pi_n \text{vol}^n
\]

$-\int * T(n,-)$

The force on $V_0$ due to the Cauchy-stress tensor is 
\[
\int_{\partial V_0} *T(n,-)
\]
Taking, for the special case of the perfect fluid, $T = - pg$, then 
\[
- \int_{\partial V_0} pi_n\text{vol}^n = -\int_{V_0} d(pi_n\text{vol}^n) = - \int \text{div}(pn) \text{vol}^n = -\int \text{grad}p \text{vol}^n
\]
using Stoke's law.  

EY : 20150720 come to think about it, we should probably treat $T$ as $T_{ij} dx^i \otimes dx^j$ and only do stuff on $dx^i$ to retain $\otimes dx^j$ to get a direction out, a 1-formed valued $m-1$ form, $\text{dim}M=m$

This is probably the correct way to think about it:

Given the Cauchy stress tensor $T = T^{ij} e_i \otimes e_j$, which is a $(2,0)$-rank tensor, $T$ is a section of the $TM\otimes TM$ bundle, i.e. $\Gamma(TM \otimes TM)$, so that $T \in \Gamma(TM \otimes TM)$.  

We want to do this:

\begin{tikzpicture}
  \matrix (m) [matrix of math nodes, row sep=2.8em, column sep=4.8em, minimum width=4.5em]
  {
\Gamma(TM\otimes TM) & \Omega^1(M,TM)                                         & \Omega^{n-1}(M,TM) \\
T= T^{ij}e_i \otimes e_j = T^{ij}e_j \otimes e_i & T^i_{ \,\,j} e^j \otimes e_i & T^i_{\,\,j} \frac{\sqrt{g}}{(n-1)!} \epsilon^j_{\,\,j_2\dots j_n}dx^{j_2}\wedge \dots \wedge dx^{j_n} \otimes e_i = T^i_{\,\,j}dS^j \otimes e_i   \\
};
  \path[->]
  (m-1-1) edge node [above] {$(\sharp,-)$} (m-1-2)
  (m-1-2) edge node [auto]  {$(*,-)$} (m-1-3);
  \path[|->]
  (m-2-1) edge node [above] {$(\sharp,-)$} (m-2-2)
  (m-2-2) edge node [auto]  {$(*,-)$} (m-2-3);
\end{tikzpicture}

Thus, the force on $V_0$ due to the Cauchy-stress tensor is 
\[
\int_{\partial V_0} T^i_{\,\,j}dS^j\otimes e_i
\]
and so for the case of $T = -pg$, 
\[
-\int_{\partial V_0} p g^i_{\,\,j} dS^j \otimes e_i
\]

For the time rate of change of momentum, see my other pdf entitled ``Aspects of Geometry in Propulsion.''  There, we find the rate of change of momentum
\[
\int_{V_0} \rho \left( \frac{\partial u^i}{ \partial t} + u^j \frac{ \partial u^i}{ \partial x^j} \right) \text{vol}^n \otimes e_i
\]

Euler's equation is then
\begin{equation}
  \rho \left( \frac{\partial u^i}{ \partial t} + u^j \frac{ \partial u^i}{ \partial x^j} \right) = - \frac{ \partial \rho }{ \partial x^j}g^{ij} 
\end{equation}



\subsection*{The energy flux}

\subsubsection*{Review of Thermodynamics}

Let $\Sigma$ be the manifold of equilibrium (and non-equilibrium) states of the system.  

\begin{proposition}[First Law: Energy Conservation]
\begin{equation}
dU = Q - dW = Q - pdV
\end{equation}
\end{proposition}
with $U, p ,V \in C^{\infty}(\Sigma)$, and $dU, Q, dW , dV \in \Omega^1(\Sigma)$, and where $U$ is internal energy, $p$ is pressure, $V$ is the volume of the system.  

\begin{proposition}[Second Law]
\begin{equation}
  Q = TdS
\end{equation} where $Q, dS \in \Omega^1(\Sigma)$, and $S \in C^{\infty}(\Sigma)$ and
with 
\begin{equation}
  dS \geq 0
\end{equation} describing irreversibility.  
\end{proposition}

\begin{definition}[Enthalpy]
  \begin{equation}
    H = U + pV
\end{equation}
where $H$ is the \textbf{enthalpy}, $H \in C^{\infty}(\Sigma)$.  
\end{definition}

Now, for $M$ being the molar mass, and defining per unit mass quantities as we go,
\[
\begin{gathered}
  dH = dU + Vdp + pdV = Q + Vdp = TdS + Vdp \\
  \xrightarrow{ 1/M} \frac{dH}{M} = dh = T \frac{dS}{M} + \frac{V}{M} dp = Tds + \frac{dp}{\rho}
\end{gathered}
\]
where $\rho = M/V$.  

Now the total energy in a fluid occupying region $V_0$ is 
\[
\int_{V_0} \left( \frac{1}{2} \rho v^2 + \rho \epsilon \right)\text{vol}^n
\]
where $\int_{V_0} \frac{1}{2} \rho v^2 \text{vol}^n$ is the total kinetic energy of fluid and $\epsilon$ internal energy per unit mass.  

The time rate of change of the energy is 
\[
\begin{gathered}
  \frac{d}{dt} \int_{V_0} ( \frac{1}{2} \rho v^2 + \rho \epsilon )\text{vol}^n = \int_{V_0} \mathcal{L}_{ \frac{\partial}{\partial t} + v } (\frac{1}{2} \rho v^2 + \rho \epsilon )\text{vol}^n = \int_{V_0} \frac{\partial }{\partial t} (\frac{1}{2} \rho v^2 + \rho \epsilon ) \text{vol}^n + \mathcal{L}_v ((\frac{1}{2} \rho v^2 + \rho \epsilon ) \text{vol}^n ) = \\
  = \int_{V_0} \frac{\partial }{\partial t} (\frac{1}{2} \rho v^2 + \rho \epsilon ) \text{vol}^n + di_v ((\frac{1}{2} \rho v^2 + \rho \epsilon ) \text{vol}^n ) 
\end{gathered}
\]

\subsection*{So-called momentum flux}

Sec. 1.7 of Landau and Lifshitz \cite{LLandauELifshitz1987}.  

Let the total momentum of a fluid in volume $V_0 \subset N$, $V_0$ a submanifold of spatial manifold $N$, with $\text{dim}V_0 = \text{dim}N =n$ be $P$:
\[
P = \int_{V_0} \rho \text{vol}^n u^i \otimes e_i \in \mathfrak{X}(M)
\]
with 
\[
\rho \text{vol}^n u^i \otimes e_i \in \Omega^n(M;TM)
\]
Now 
\[
\dot{P} := \frac{d}{dt}P = \int_{V_0} \mathcal{L}_{\frac{ \partial}{\partial t} + u} (\rho \text{vol}^nu^i \otimes e_i)
\]
which can be shown (see Yeung (2015) ``Aspects of Geometry in Propulsion'') to be
\[
\dot{P} = \int_{V_0} \frac{ \partial (\rho u^i)}{ \partial t} \text{vol}^n \otimes e_i + \int_{\partial V^0} \rho u^i i_u \text{vol}^n \otimes e_i
\]

Now
\[
\rho u^i i_u \text{vol}^n = \rho u^i \frac{\sqrt{g}}{(n-1)!} \epsilon_{ki_2\dots i_n} u^k dx^{i_2} \wedge \dots \wedge dx^{i_n} = \rho u^i u^k \frac{\sqrt{g}}{ (n-1)!} \epsilon_{ki_2 \dots i_n} dx^{i_2} \wedge \dots \wedge dx^{i_n} = \rho u^i u^k dS_k
\]

So starting from equating the time rate of change of momentum $\dot{P}$ with the external forces on it, $\int_{\partial V_0} T^i_{\,\,j}dS^j \otimes e_i$, then, for the special case of the perfect fluid, $T = -pg$
\[
\begin{gathered}
  \dot{P} = \int_{\partial V_0} T^i_{\,\,j}dS^j \\ 
  \Longrightarrow \int_{V_0} \frac{ \partial \rho u^i}{ \partial t} \text{vol}^n \otimes e_i + \int_{\partial V^0} \rho u^i u^k dS_k \otimes e_i = - \int_{\partial V^0} p g^{ij} dS_j \otimes e_i
\end{gathered}
\]
Now move the boundary term on the left hand side, $\int_{\partial V^0} \rho u^i u^j dS_j \otimes e_i$ over to the right hand side:
\[
\begin{gathered}
\int_{V_0} \frac{ \partial \rho u^i}{ \partial t} \text{vol}^n \otimes e_i  = -  \int_{\partial V^0} \rho u^i u^k dS_k \otimes e_i - \int_{\partial V^0} p g^{ij} dS_j \otimes e_i = -\int_{\partial V_0} (\rho u^i u^j + p g^{ij}) dS_j \otimes e_i
\end{gathered}
\]
Then the momentum flux tensor $\Pi \in \Gamma(TM\otimes TM)$, in this case, takes the form
\[
\Pi^{ij} = \rho u^i u^j + p g^{ij}
\]

\subsection*{Drag force in potential flow past a body}

Sec. 1.11 of Landau and Lifshitz \cite{LLandauELifshitz1987}.  

The dictionary: for $v = v^i \frac{\partial }{ \partial x^i} \in \mathfrak{X}(M)$, \\
$v^{\flat} = v_i dx^i = g_{ij} v^j dx^i \in \Omega^1(M)$

\[
\begin{gathered}
  *v^{\flat} = \frac{ \sqrt{g}}{(n-1)!} v_i \epsilon^i_{ \,\, j_2 \dots j_n} dx^{j_2} \wedge \dots \wedge dx^{j_n} \\ 
  d*v^{\flat} = \frac{ \partial }{ \partial x^k} (\sqrt{g} v_i) \frac{ \epsilon^i_{ \,\, j_2 \dots j_n} }{(n-1)!} dx^k \wedge dx^{j_2} \wedge \dots \wedge dx^{j_n} = \text{div}(v) \text{vol}^n + \frac{ \partial \ln{ \sqrt{g}}}{ \partial x^i} v^i \text{vol}^n \\ 
  dv^{\flat} = \frac{ \partial v_j}{ \partial x^j} dx^i \wedge dx^j \\ 
  *dv^{\flat} = \frac{ \partial v_j}{ \partial x^i} \frac{ \sqrt{g}}{ (n-2)!} \epsilon^{ij}_{ \,\, k_3 \dots k_n} dx^{k_3} \wedge \dots \wedge dx^{k_n} 
\end{gathered}
\]
if $\sqrt{g} =1$, $n=3$, 
\[
(*dv^{\flat})^{\sharp} = \text{curl}(v)
\]

If $dv^{\flat}=0$, then $v^{\flat} = d\phi$ (i.e. $v^{\flat} $ is an exact form).  


\section{Viscous Fluids} Chapter 2 of Landau and Lifshitz \cite{LLandauELifshitz1987}

Introduce a viscous stress tensor $\sigma' \in \Gamma(TM\otimes TM)$.  Then do the following transformations:

\begin{tikzpicture}
  \matrix (m) [matrix of math nodes, row sep=3.8em, column sep=4.8em, minimum width=2.2em]
  {
\sigma' & (\sigma')^i_{\,\,j} e^j \otimes e_i & (\sigma')^{ij} dS_j \otimes e_i    \\
};
  \path[->]
  (m-1-1) edge node [above] {$(\sharp,-)$} (m-1-2)
  (m-1-2) edge node [auto]  {$(*,-)$} (m-1-3)
  ;
\end{tikzpicture}

So the external force on the fluid in region $V_0$ is 
\[
\int_{\partial V_0} (T^{ij} + (\sigma')^{ij}) dS_j \otimes e_i
\]


\begin{tikzpicture}
  \matrix (m) [matrix of math nodes, row sep=3.8em, column sep=4.8em, minimum width=2.2em]
  {
\sigma & \tau \\
F(\sigma) & F(\tau) \\
};
  \path[->]
  (m-1-1) edge node [above] {$c$} (m-1-2)
          edge node [auto]  {$F$} (m-2-1)
  (m-1-2) edge node [auto]  {$F$} (m-2-2)
  (m-2-1) edge node [above] {$F(c)$} (m-2-2)        
  ;
\end{tikzpicture} \quad \quad \quad \,  \begin{tikzpicture}
  \matrix (m) [matrix of math nodes, row sep=3.8em, column sep=4.8em, minimum width=2.2em]
  {
\rho & \sigma & \tau \\
F(\rho) & F(\sigma) & F(\tau) \\ 
};
  \path[->]
  (m-1-1) edge node [above] {$c_1$} (m-1-2)
  edge[bend left=45] node [above] {$c_2\circ c_1$} (m-1-3)
  edge node [auto] {$F$} (m-2-1)
  (m-1-2) edge node [above] {$g$} (m-1-3)
  edge node [auto] {$F$} (m-2-2)
  (m-1-3) edge node [auto] {$F$} (m-2-3)
  (m-2-1) edge node [above] {$F(c_1)$} (m-2-2)
  edge[bend right=45] node [below] {$F(c_2\circ c_1) = F(c_2) \circ F(c_1)$} (m-2-3)
  (m-2-2) edge node [above] {$F(c_2)$} (m-2-3)  
;
\end{tikzpicture} 



\section{Navier-Stokes Equations}

I am following closely Chorin and Marsden, Sec. 1.3 ``The Navier-Stokes Equations'', Ch. 1 The Equations of Motion \cite{AChorinJMarsden2000}.

``On the left hand side,''
\[
\begin{gathered}
  \frac{d}{dt} \int_{B(t)} \rho u \text{vol}^n =: \frac{d}{dt} P := \frac{d}{dt} \int_{B(t)} m \otimes u = \int_{B(t)} ( \mathcal{L}_{ \frac{ \partial }{ \partial t} +u } m ) \otimes u + \int_{B(t)} m \otimes \mathcal{L}_{\frac{\partial}{\partial t} + u } u = \\
  = 0 + \int_{B(t)} m \otimes \left( \frac{ \partial u}{ \partial t} + u^i \frac{ \partial u^j}{ \partial x^i} \frac{ \partial }{ \partial x^j} \right) = \int_{B(t)} m \otimes \left( \left(\frac{ \partial u^j}{ \partial t} + u^i \frac{ \partial u^j}{ \partial x^i} \right) \frac{ \partial }{ \partial x^j} \right)
\end{gathered}
\]
assuming mass conservation $\mathcal{L}_{\frac{\partial }{\partial t} + u } m =0$.  

``On the right hand side'' are the physical forces on the fluid.  Consider first only the forces on its surface $\partial B(t)$:
\[
\int T^{ij}dS_j \otimes \frac{\partial}{\partial x^i}
\]
with $T$ being the stress tensor.  For the case of a fluid, 
\[
T = -gp + \sigma \in \Gamma(\otimes^2TM)
\]
where $\sigma$ is usually called the \emph{viscous stress tensor}.  

For deformation tensor $E$, $\sigma$ can appear in two forms for its constitutive relation with $E$ (this constitutive relation is assumed to be linear, $E$ and $\sigma$ are symmetric rank-2 tensors, due to angular momentum conservation, and fluid is assumed to be isotropic), 
\[
\begin{gathered}
  \sigma = \lambda ( \text{tr}E)1 + 2 \mu E \\ 
  \sigma = 2 \mu (E - \frac{1}{d}\text{tr}(E) 1) + \eta \text{tr}(E) 1 \in \Gamma(\otimes^2 TM)
\end{gathered}
\]
with $\mu$ being the 1st. coefficient of viscosity \\
\phantom{with }$\eta =  \lambda + \frac{2}{d} \mu$ being the 2nd. coefficient of viscosity.  

Now if we took the exterior derivative $d$ of each of the following:
\[
\begin{gathered}
  -g^{ij} p dS_j = -p g^{ij} \frac{\sqrt{g}}{(n-1)!} \epsilon_{ji_2 \dots i_n} dx^{i_2} \wedge \dots \wedge dx^{i_n} \xrightarrow{ d } -\frac{1}{\sqrt{g}}\frac{ \partial (p g^{ik} \sqrt{g})}{ \partial x^k} \text{vol}^n
\end{gathered}
\]
Consider $E = \frac{1}{2} \left( g_{kj} \frac{ \partial u^k}{ \partial x^i} + g_{ik} \frac{ \partial u^k}{ \partial x^j} + \frac{ \partial g_{ij}}{ \partial x^k} u^k \right)dx^i \otimes dx^j \in \Gamma(\otimes^2 T^*M )$, then
\[
E^{\sharp} =  \frac{1}{2} \left( g_{kj} \frac{ \partial u^k}{ \partial x^i} + g_{ik} \frac{ \partial u^k}{ \partial x^j} + \frac{ \partial g_{ij}}{ \partial x^k} u^k \right) g^{il} \frac{ \partial }{ \partial x^l} \otimes g^{jm} \frac{ \partial }{ \partial x^m} = \frac{1}{2} \left( g^{il} \frac{ \partial u^m}{ \partial x^i} + g^{jm} \frac{ \partial u^l }{ \partial x^j} + g^{il} g^{jm} \frac{ \partial g_{ij} }{ \partial x^k} u^k \right) \frac{\partial}{ \partial x^l} \otimes \frac{\partial }{ \partial x^m}
\]
\[
\begin{gathered}
  \Longrightarrow \frac{ \partial (E^{\sharp})^{ij} }{ \partial x^j} = \\
  = \frac{1}{2} \left( \frac{ \partial g^{ki} }{ \partial x^j} \frac{ \partial u^j}{ \partial x^k} + g^{ki} \frac{ \partial^2 u^j }{ \partial x^j \partial x^k } + \frac{ \partial g^{kj} }{ \partial x^j} \frac{ \partial u^i }{ \partial x^k} + g^{kj} \frac{ \partial^2 u^i }{ \partial x^j \partial x^k} + \frac{ \partial }{ \partial x^j } (g^{li} g^{mj} ) \frac{ \partial g_{lm}}{ \partial x^k} u^k + g^{li} g^{mj} \frac{ \partial^2 g_{lm}}{ \partial x^j \partial x^k} u^k + g^{li} g^{mj} \frac{\partial g_{lm}}{ \partial x^k} \frac{\partial u^k}{ \partial x^j} \right)
\end{gathered}
\]
If $g^{ij} = \delta^{ij}$ (i.e. Cartesian coordinates)
\[
\begin{gathered}
  (E^{\sharp})^{ij} = \frac{1}{2} \left( \frac{ \partial u^j}{ \partial x^i} + \frac{ \partial u^i }{ \partial x^j} \right) \\ 
  \Longrightarrow \text{tr}(E^{\sharp}) = \frac{ \partial u^i}{ \partial x^i}
\end{gathered}
\]
\[
\begin{gathered}
  \frac{ \partial (E^{\sharp})^{ij}}{ \partial x^j} = \frac{1}{2} \left( \frac{ \partial^2 u^j}{ \partial x^j \partial x^i} + \frac{ \partial^2 u^i }{ \partial x^j  \partial x^j } \right) := \frac{1}{2} \left( \frac{ \partial }{ \partial x^i} (\text{div}u) + \Delta u^i \right)
\end{gathered}
\]
So then, if $\lambda$, $\mu$ don't depend on position, for 
\[
\begin{gathered}
  \sigma^{ij} dS_j = \lambda (\text{tr}E) \delta^{ij} + 2\mu E^{ij} \xrightarrow{ d} \left( \frac{ \partial ( \lambda (\text{tr}E ) \delta^{ij }\sqrt{g} )}{ \partial x^j} + 2 \frac{ \partial ( \mu E^{ij} \sqrt{g}) }{ \partial x^j} \right)\text{vol}^n \frac{1}{\sqrt{g}}= \\
  = \left( \lambda \frac{ \partial (\text{tr}E\sqrt{g})}{ \partial x^i } + 2\mu \frac{ \partial E^{ij}\sqrt{g}}{ \partial x^j } \right) \text{vol}^n \frac{1}{\sqrt{g}}
\end{gathered}
\]
then with $g^{ij} = \delta^{ij}$,
\[
\begin{gathered}
  d \sigma^{ij} dS_j = \left( \lambda \frac{ \partial }{ \partial x^i} \text{div}u + \mu \left( \frac{\partial }{ \partial x^i} (\text{div}u ) + \Delta u^i \right) \right)\text{vol}^n
\end{gathered}
\]

Thus, we reproduce, in Cartesian coordinates, the \emph{Navier-Stokes equations} for \emph{compressible, viscous} fluid flow:
\begin{equation}
\boxed{
  \rho \left( \frac{ \partial u^i}{ \partial t} + u^j \frac{ \partial u^i}{ \partial x^j} \right) = - \frac{ \partial p}{ \partial x^i} + (\lambda +  \mu ) \frac{ \partial}{\partial x^i} \text{div}u + \mu \Delta u^i
}
\end{equation}

The case of \emph{incompressible homogeneous} flow is when $\rho = \rho_0$, a constant, and $\text{div}u=0$ (i.e. ``no volume expansion''), so that the Navier-Stokes equations for incompressible flow is 
\[
\rho_0 \left( \frac{ \partial u^i}{ \partial t} + u^j \frac{ \partial u^i}{ \partial x^j} \right) = -\frac{ \partial p }{ \partial x^i} + \mu \Delta u^i
\]

Set $L$ \emph{characteristic length} \\
\phantom{Set }$U$ \emph{characteristic velocity}

Then
\[
\begin{aligned}
  & (u')^i = \frac{u^i}{U} \\ 
  & (x')^j = \frac{x^j}{L} \\ 
  & t' = \frac{t}{ L/U}
\end{aligned} \quad \quad \quad \, \begin{aligned}
  & \frac{ \partial u^i}{ \partial t} = \frac{U}{ L/U} \frac{ \partial (u^i)' }{ \partial t'} = \frac{U^2}{L} \frac{ \partial (u^i)' }{ \partial t' } \\ 
  & \frac{ \partial u^i }{ \partial x^j} = \frac{U}{L} \frac{ \partial (u')^i }{ \partial (x')^j } \\
  & \frac{ \partial u^i}{ \partial t} + u^j \frac{ \partial u^i}{ \partial x^j} = \frac{U^2}{L} \frac{ \partial (u^i)' }{ \partial t' } + U (u')^j \frac{U}{L} \frac{ \partial (u')^i }{ \partial (x')^j } = \frac{U^2}{L} \left( \frac{ \partial (u^i)' }{ \partial t' } + (u')^j \frac{ \partial (u')^i }{ \partial (x')^j } \right)
\end{aligned}
\]
\[
\begin{gathered}
  \begin{aligned}
    & \text{div}u = \frac{L}{U} \text{div}u' \\ 
    & \Delta (u')^i = \frac{L^2}{U} \Delta u^i
  \end{aligned} \quad \quad \quad \, \begin{aligned} 
    & - \frac{ \partial p}{ \partial x^i} = \frac{-1}{L} \frac{ \partial p}{ \partial (x')^i } \xrightarrow{ \cdot \frac{L}{ \rho U^2} } -\frac{1}{ \rho U^2} \frac{ \partial p }{ \partial (x')^i } \\
    & (\lambda + \mu) \frac{ \partial }{ \partial x^i} \text{div}u = (\lambda + \mu) \frac{U}{L^2} \frac{ \partial }{ \partial (x')^i } \text{div}u' \xrightarrow{ \frac{L}{ \rho U^2} } (\lambda + \mu ) \frac{1}{ \rho LU } \frac{ \partial }{ \partial (x')^i } \text{div}u' \\
    & \mu \Delta u^i = \mu \frac{U}{ L^2} \Delta (u')^i \xrightarrow{ \frac{L}{ \rho U^2} } \frac{ \mu L}{ \rho U^2} \Delta u^i = \frac{ \mu }{ \rho LU} \Delta(u')^i
\end{aligned}
\end{gathered}
\]
Define $R:= \frac{ \rho LU }{ \mu }$, the \textbf{Reynolds number}.  $R$ is a dimensionless quantity.  $\frac{(\lambda + \mu )}{ \rho L U}$ is a dimensionless quantity.  

cf. Sec. 19 The law of similarity, Landau and Lifshitz \cite{LLandauELifshitz1987}:

For
\begin{equation}
  \text{Re} := \frac{ \rho L U }{ \mu }
  \end{equation}
Then the velocity distribution is
\[
\mathbf{u} = \mathbf{u}(t,\mathbf{x}) = Uf\left( \frac{ \mathbf{x} }{L}, \text{Re} \right)
\]

Then pressure distribution is
\[
p = \rho U^2 f\left( \frac{\mathbf{x}}{L}, \text{Re} \right)
\]

For a drag force acting on the body,
\[
F_{\text{drag}} = \rho U^2 L^2 f(\text{Re}) \qquad \, \text{ caveat }
\]

If body force (e.g. gravity), consider the \emph{Froude number} $\text{Fr}$,
\begin{equation}
  \text{Fr} = \frac{U^2 }{ L g}
  \end{equation}

For nonsteady flow, then consider some time interval $T$, characteristic of flow: e.g. oscillations, $T \equiv $ period of oscillation
\[
\text{St}  = \frac{ UT}{L} \qquad \, \text{ caveat }
\]
If the oscillations occur spontaneously (i.e. not under external exciting force), then the oscillation $\phi$ will be a definite function of $\text{Re}$:
\[
\phi = f(\text{Re})
\]



Bhatia, Norgard, Pascucci, Bremer have an insightful survey on the so-called Helmholtz-Hodge decomposition and expands upon it and what's out there already in the literature \cite{HBhatiaGNorgardVPascucciPBremer2013}.  





Let domain be the smooth submanifold $\mathcal{D} \subset N$, $N$ is the spatial manifold. $\text{dim}a = \text{dim}N=n$; spacetime manifold $M = \mathbb{R}\times N$.  \\
\phantom{Let} $a =(a^i) \in \mathcal{D}$ is a particle label.  


It'd be instructive to compare expressions between Chorin and Marsden \cite{AChorinJMarsden2000} and Landau and Lifshitz \cite{LLandauELifshitz1987}:
\[
\begin{gathered}
  \begin{gathered}
    \sigma = 2 \mu ( E - \frac{1}{d} \text{tr}(E)1 ) + \eta \text{tr}(E) 1 \in \Gamma(\otimes^2TM) \\ 
    E = \frac{1}{2} ( g_{kj} \frac{ \partial u^k}{ \partial x^i} + g_{ik} \frac{ \partial u^k}{ \partial x^j} + \frac{ \partial g_{ij}}{ \partial x^k} u^k ) dx^i \otimes dx^j \in \Gamma(\otimes^2 T^*M) \\ 
    E^{\sharp} = \frac{1}{2} \left( \frac{ \partial u^j}{ \partial x^i} + \frac{ \partial u^i}{ \partial x^j} \right) \frac{ \partial }{ \partial x^i} \otimes \frac{ \partial x^j} \in \Gamma(\otimes^2 TM) \\
\text{tr}E^{\sharp} = \frac{ \partial u^i}{ \partial x^i}
\end{gathered} \Longleftrightarrow \begin{gathered}
    \sigma'_{ik} = \eta \left( \frac{ \partial v_i}{ \partial x_k} + \frac{ \partial v_k}{ \partial x_i} - \frac{2}{3} \delta_{ik} \frac{ \partial v_l}{ \partial x_l } \right) +\eta \delta_{ik} \frac{ \partial v_l}{ \partial x_l}
\end{gathered} \\
\begin{aligned}
&  \mu  & \equiv \text{1st. coefficient of viscosity } \\ 
&  \eta = \lambda + \frac{2}{d} \mu & \equiv \text{ 2nd. viscosity }
\end{aligned} \Longleftrightarrow \begin{aligned}
&  \eta & \equiv \text{ 1st. coefficient of viscosity }\\
& \zeta & \equiv \text{ second viscosity } \end{aligned} 
\end{gathered}
\]
\[
\begin{gathered}
\begin{gathered}
  \rho \left( \frac{ \partial u^i}{ \partial t} + u^j \frac{ \partial u^i }{ \partial x^j} \right) = - \frac{ \partial p}{ \partial x^i } + (\lambda + \mu) \frac{ \partial }{ \partial x^i} \text{div}u + \mu \Delta u^i = \\
  = - \frac{ \partial p}{ \partial x^i} + \mu \Delta u^i + \left( \eta + (1- \frac{2}{d} ) \mu \right) \frac{ \partial }{ \partial x^i} \text{div}u
\end{gathered} \Longleftrightarrow \begin{gathered}
\rho \left[ \frac{ \partial v}{ \partial t} + (v\cdot \text{grad})v \right] = -\text{grad}p + \eta \Delta v + \left(\zeta + \frac{ \eta}{3} \right)\text{grad}\text{div}v 
\end{gathered}
\end{gathered}
\]

\subsection{Material Derivative}

Consider $u^j \frac{ \partial u^i}{ \partial x^j}$ or $(u\cdot \text{grad})u$.  Landau and Lifshitz on pp. 48, Sec. 15 has a useful table of the Equations of Motion in Curvilinear Coordinates \cite{LLandauELifshitz1987}.  Recall the metric $g$ for cylindrical and spherical coordinates, for $g\in \otimes^2 T^*M$ and $g^{-1} \in \otimes^2 TM$

\[
\begin{gathered}
  g = dr^2 + r^2 d\phi^2 + dz^2 \\
  g^{-1} = (\frac{\partial}{\partial r})^2 + \frac{1}{r^2} (\frac{ \partial }{ \partial \phi })^2 + ( \frac{ \partial }{ \partial z} )^2
\end{gathered} \quad \quad \quad \, 
\begin{gathered}
  g = d\rho^2 + \rho^2 d\theta^2 + \rho^2 \sin{\theta} d\phi^2  \\
  g^{-1} = (\frac{\partial}{\partial \rho})^2 + \frac{1}{\rho^2} (\frac{ \partial }{ \partial \theta })^2 + \frac{1}{ \rho^2 \sin{\theta}}( \frac{ \partial }{ \partial \phi} )^2
\end{gathered}
\]

\section{Energy transport and Energy dissipation}

Chorin and Marsden, pp. 10 \cite{AChorinJMarsden2000}.

\begin{definition}
  Fluid is \emph{incompressible} if $\frac{d}{dt} \int \text{vol}^n =0$
\end{definition}

Now 
\[
\begin{gathered}
  \frac{d}{dt} \int \text{vol}^n = \int \mathcal{L}_{ \frac{ \partial }{ \partial t} + u } \text{vol}^n = \int \frac{ \partial }{ \partial t} \text{vol}^n + \mathcal{L}_u \text{vol}^n = \int 0 + di_u \text{vol}^n + i_u d\text{vol}^n = \int d \frac{\sqrt{g}}{ n } \epsilon_{i_1 i_2 \dots i_n} u^{i_1 } dx^{i_2} \wedge \dots \wedge dx^{i_n} + 0 = \\
  = \int \frac{1}{\sqrt{g}} \frac{ \partial (\sqrt{g} u^k )}{ \partial x^k} \text{vol}^n 
\end{gathered}
\]
If $\frac{d}{dt} \int \text{vol}^n =0$ i.e. fluid is incompressible, $\text{div}u := \frac{1}{\sqrt{g}} \frac{ \partial (\sqrt{g} u^k  )}{ \partial x^k } = 0 $  

Let 
\[
E_{\text{KE}} := \frac{1}{2} \int_{B(t)} \rho |u|^2 \text{vol}^n
\]
Then
\[
\begin{gathered}
  \frac{d}{dt} E_{\text{KE}} = \frac{d}{dt} \left[ \frac{1}{2} \int_{B(t)} \rho |u|^2 \text{vol}^n \right] = \frac{1}{2} \int_{B(t)} \mathcal{L}_{\frac{\partial}{\partial t} + u }( m|u|^2) = \frac{1}{2} \int_{B(t)} ( \mathcal{L}_{\frac{\partial }{ \partial t} + u } m ) |u|^2 + m \mathcal{L}_{\frac{ \partial }{ \partial t} + u } |u|^2  = \frac{1}{2} \int_{B(t)} 0 + m \mathcal{L}_{\frac{\partial}{\partial t} +u } |u|^2 
\end{gathered}
\]
Now
\[
\frac{\partial}{\partial t} |u|^2  = u^i \frac{\partial}{\partial t} (g_{ij} u^j ) + u_i \frac{\partial u^i}{ \partial t} 
\]
If $g_{ij}$ is time-independent, 
\[
\frac{\partial }{ \partial t} |u|^2 = 2u_i \frac{\partial u^i}{\partial t}
\]
Also,
\[
\begin{gathered}
  \mathcal{L}_u |u|^2  = u^k \frac{\partial }{ \partial x^k} (g_{ij} u^j u^i) = 2u^k u_i \frac{\partial u^i}{ \partial x^k} + u^k u^j u^i \frac{ \partial g_{ij} }{ \partial x^k} = 2u_i ( u^k \frac{ \partial u^i }{ \partial x^k} + \frac{1}{2}u^l \frac{ \partial g_{jk}}{ \partial x^l} u^j g^{ik} ) = \\
  = 2u_i \left( u^k \frac{ \partial u^i}{ \partial x^k} + \Gamma^i_{jk} u^j u^k \right) = 2u \cdot \nabla_u u
\end{gathered}
\]
and all (I claim; I worked it out) that was required was that $g$ be metric-compatible, i.e. $\nabla g=0$.  

Thus,
\[
\frac{d}{dt} E_{\text{KE}} =  \int_{B(t)} m u_i \left(  \frac{ \partial u^i }{ \partial t} + \nabla_u u^i \right)
\]

Consider the ``right hand side'' (RHS), the work done on the fluid system by the stresses on the surface boundary, $T \in \Gamma( \otimes^2 T^*M)$, and body forces per mass, $b$.  I propose that this work takes this form:
\begin{equation}
  \int_{\partial B} u_i T^{ij} dS_j + \int_B m b^i u_i \equiv \int_{\partial B} T(u,dS) + \int_B m \langle u,b \rangle
\end{equation}

Now
\[
\begin{gathered}
\int_{\partial B} u_i T^{ij} dS_j = \int_B \frac{1}{\sqrt{g}} \frac{ \partial ( u_i T^{ij} \sqrt{g})}{ \partial x^j} \text{vol}^n \equiv \int_B \text{div}(u_i T^{ij}) \text{vol}^n = \int_B \left( u_i \frac{ \partial T^{ij}}{ \partial x^j} + T^{ij} \frac{1}{\sqrt{g}} \frac{ \partial (u_i \sqrt{g})}{ \partial x^j} \right) \text{vol}^n \equiv \\
\equiv \int_B \left( u_i \frac{ \partial T^{ij}}{ \partial x^j} + T^{ij} \text{div}u_i \right)\text{vol}^n
\end{gathered}
\]

Suppose $T^{ij} = -p g^{ij}$.  Then
\[
\begin{gathered}
  \int_B \text{div}(u_i T^{ij})\text{vol}^n \equiv \int_B \frac{-1}{\sqrt{g}} \frac{ \partial (u_i pg^{ij} \sqrt{g})}{ \partial x^j} \text{vol}^n = -\int_B \frac{1}{\sqrt{g}} \frac{ \partial (u^j p \sqrt{g})}{ \partial x^j} \text{vol}^n = -\int_B \left( u^j \frac{ \partial p}{ \partial x^j} + \frac{p}{ \sqrt{g}} \frac{ \partial (u^j \sqrt{g})}{ \partial x^j} \right) \text{vol}^n = \\
  = -\int_B \left( u^j \frac{ \partial p }{ \partial x^j } + p\text{div}u \right) \text{vol}^n
\end{gathered}
\]
For $T^{ij} = -pg^{ij}$, this result is the most general case.  If the fluid is incompressible, then $\text{div}u=0$, so the contribution to the work done on the fluid is $-\int_B u^j \frac{ \partial p }{ \partial x^j} \text{vol}^n$.  

Thus, energy conservation, for an incompressible fluid is 
\[
 \rho u_i \left( \frac{ \partial u^i }{ \partial t} + \nabla_u u^i \right) = - u^j \frac{ \partial p}{ \partial x^j }+ \rho u_i b^i
\]


\part{Notes on Sabersky}



\section{Shear: shear stress}

cf. Chapter 1 Introduction

In Appendix A Forms in Continuum Mechanics, Subsection A.g. Some Typical Computations Using Forms, Frankel (2004) \cite{TFrankel2004} defines the so-called \textbf{rate of deformation} tensor, which Frankel describes as measuring how the flow deforms the physical system in consideration.  

Consider curve $\begin{aligned} & \quad \\ 
  & \gamma : \mathbb{R} \to N \\
  & \gamma(t) \in N \end{aligned}$ and generating velocity vector field $u = \dot{\gamma} \in \mathfrak{X}(N)$.  

Now $\begin{aligned} & \quad \\
  & \gamma(0)= x(0) \\
  & \gamma(t) = x(t) \end{aligned}$ and so we recall the idea of $\psi_t$, a local 1-parameter group $(\psi_t)_{t\in I \subset \mathbb{R}}$ of local diffeomorphisms s.t. $\psi_t(x(0)) = x(t) \in N$.  

Now recall the definition of the Lie derivative of a tensor (e.g. Def. 2.2.6 on pp. 58, Chapter 2 Lie Groups and Vector Bundles of Jost (2011) \cite{JJost2011}: 
\[
\mathcal{L}_ug := \frac{d}{dt} \left. (\psi_t^* g) \right|_{t=0} = \lim_{t\to 0 } \frac{ \psi_t^*g' - g}{t} = \lim_{t\to 0} \frac{ \left( g'_{ij} \frac{ \partial x^i}{ \partial a^k } \frac{ \partial x^k}{ \partial a^l } - g_{kl} \right) }{ t} da^k \otimes da^l
\]
If $\mathcal{L}_ug = 0$, then metric $g$ doesn't change along integral curves of $u$, so $u$ generates cont. 1-parameter family of symmetries $\psi_t$ for $g$.  \footnote{Fionn Fitzmaurice. ``Differential Geometry.''  \url{https://www.maths.tcd.ie/~fionn/dg/dg.pdf} }

Otherwise, $\mathcal{L}_ug\neq 0$.  

Then $u$ is \emph{not} a Killing vector.  $\psi_t$ are not symmetric diffeomorphisms (i.e. isometries).  

Let's compute $\mathcal{L}_ug$.  

Calin and Chang (2005) \cite{OCalinDChang2005} does a great job in calculating out \emph{explicitly} with \emph{explicit} calculation steps in all their proofs of theorems, lemmas, claims.  For instance, in Chapter 5 Conservation Theorems, Section 5.3 The Energy-momentum tensor, Subsection 5.3.4 Divergence of the energy-momentum tensor, pp. 83, Lemma 5.33 of Calin and Chang (2005) \cite{OCalinDChang2005}, it gives the explicit calculation of $\mathcal{L}_ug$, exactly what we need:
\[
\begin{gathered}
  \mathcal{L}_ug = ( \mathcal{L}_u g_{ij} dx^i ) \otimes dx^j + g_{ij} dx^i \otimes \mathcal{L}_u dx^j = (i_u d(g_{ij} dx^i ) + d(g_{ij}u^i )) \otimes dx^j + g_{ij}dx^i \otimes du^j = \\ 
   = (i_u \left( \frac{ \partial g_{ij}}{ \partial x^k} dx^k \wedge dx^i \right) + \frac{ \partial (g_{ij}u^i )}{ \partial x^k} dx^k ) \otimes dx^j + g_{ij} dx^i \otimes \frac{ \partial u^j}{ \partial x^k} dx^k  = \\
  = \left( u^k \frac{ \partial g_{ij}}{ \partial x^k} dx^i - u^i \frac{ \partial g_{ij} }{ \partial x^k} dx^k + u^i \frac{ \partial g_{ij}}{ \partial x^k } dx^k + g_{ij} \frac{ \partial u^i }{ \partial x^k} dx^k \right) dx^j + g_{ij} \frac{ \partial u^j }{ \partial x^k} dx^i \otimes dx^k = \left( u^k \frac{ \partial g_{ij}}{ \partial x^k} + g_{kj} \frac{ \partial u^k}{ \partial x^i } + g_{ik} \frac{ \partial u^k}{ \partial x^j} \right) dx^i \otimes dx^j
\end{gathered}
\]
This calculation is completely equivalent to using, besides the product rule on symmetric tensors, the $\mathbb{K}$-linearity property of $\mathcal{L}$,
\begin{equation}\label{Eq:Liederivativeofthemetric}
\begin{gathered}
  \mathcal{L}_u g = (\mathcal{L}_u g_{ij}) dx^i \otimes dx^j + g_{ij} (( \mathcal{L}_u dx^i \otimes dx^j ) + dx^i \otimes \mathcal{L}_u dx^j ) = (u^k \frac{ \partial g_{ij}}{ \partial x^k} + g_{kj} \frac{ \partial u^k}{ \partial x^i} + g_{ik} \frac{ \partial u^k }{ \partial x^j} ) dx^i \otimes dx^j
\end{gathered}
\end{equation}
Now, by theorem, for any smooth manifold, $\exists \, !$ Levi-Civita connection that's torsion-free (which can always be constructed to be torsion-free) and metric-compatible (which has to be defined, but otherwise, we've got not enough structure).  Then on any coordinate chart, since by requiring metric compatibility, then $\nabla g = 0$, so
\[
\begin{gathered}
  \nabla g= 0 \Longrightarrow \\ 
  \frac{ \partial g_{ij}}{ \partial x^k} - g_{lj} \Gamma^l_{ \; \; ik } - g_{il } \Gamma^l_{ \; \; jk} = \frac{ \partial g_{ij} }{ \partial x^k} - g_{lj} \Gamma^l_{ \; \; ki} - g_{il } \Gamma^l_{ \; \; kj} = 0 
\end{gathered}
\]
using torsion free $\nabla$ (so $\Gamma^l_{ \; \; ki} = \Gamma^l_{ \; \; ik}$ i.e. index symmetry) property.    

Then from our expression for $\mathcal{L}_ug$, Eq. \ref{Eq:Liederivativeofthemetric}, 
\[
\begin{gathered}
  u^k \frac{ \partial g_{ij}}{ \partial x^k} + g_{kj} \frac{ \partial u^k}{ \partial x^i} + g_{ik} \frac{ \partial u^k }{ \partial x^j}   = u^k \frac{ \partial g_{ij}}{ \partial x^k} + g_{kj} ((\nabla_iu)^k - \Gamma^k_{ \; \; ji} u^j ) + g_{ik} ((\nabla_j u)^k - \Gamma^k_{ \; \; ij} u^i ) = \\
  = \underbrace{ u^k \frac{ \partial g_{ij} }{ \partial x^k} - g_{kj} \Gamma^k_{ \; \; ji} u^j - g_{ik} \Gamma^k_{ \; \; ij} u^i }_{ = \nabla_u g = \nabla g(u) = 0 } + g_{kj} (\nabla_i u)^k + g_{ik} (\nabla_j u)^k 
\end{gathered}
\]

Thus,
\begin{equation}\label{Eq:Liederivativeofmetricresult}
  \boxed{ \mathcal{L}_u g = (g_{kj} (\nabla_i u)^k + g_{ik} (\nabla_j u)^k ) dx^i \otimes dx^j }
\end{equation}
for the Levi-Civita connection on any metric manifold $(N,g)$.  

Note that wikipedia calls this the ``strain rate'' tensor in its ``Viscous stress tensor'' article.  

The viscous stress tensor, $\epsilon_{ij}$ or $\epsilon = \epsilon_{ij} dx^i \otimes dx^j \in \Gamma( \otimes^2 T^*M)$, then is related, for the case of Newtonian fluids, to $\mathcal{L}_ug$ in the following manner:
\[
\epsilon_{ij} = \mu_{ij}^{ \; \; kl} (\frac{1}{2} \mathcal{L}_ug)_{kl} \text{ or } \epsilon = (\mu_{ij}(\frac{1}{2} \mathcal{L}_ug) ) dx^i \otimes dx^j = \mu(\frac{1}{2} \mathcal{L}_u g)
\]
where $\otimes^2 T^*M \xrightarrow{ \mu } \otimes^2 T^*M$.  

Examples, i.e. specific cases.  

Suppose $u$ is all in $\frac{ \partial }{ \partial x}$ direction (1 direction).  
\[
u = u^x \frac{ \partial }{ \partial x} \equiv u \frac{ \partial }{ \partial x}
\]
\[
\begin{gathered}
  g_{kj} (\nabla_i u)^k = g_{kj} \left( \frac{ \partial u^k}{ \partial x^i} + \Gamma^k_{ \; \; ji} u^j \right) = g_{xj} \left( \frac{ \partial u}{ \partial x^i} \right) + g_{kj} \Gamma^k_{ \; \; xi } u \\
  \mathcal{L}_u g = (g_{kj} (\nabla_i u)^k + g_{ik} (\nabla_j u)^k ) dx^i \otimes dx^j = (g_{xj} \left( \frac{ \partial u}{ \partial x^i} \right) + g_{kj} \Gamma^k_{ \; \; xi} u + g_{xi} \left( \frac{ \partial u }{ \partial x^j} \right) + g-{ki} \Gamma^k_{ \; \; xj} u ) dx^i \otimes dx^j = \\
  = 2 \left( g_{xj} \left( \frac{ \partial u}{ \partial x^i} \right)+ g_{ki} \Gamma^k_{ \; \; xj} u \right) dx^i \otimes dx^j
\end{gathered}
\]
if $u=u(y)$, $\Gamma^k_{ \; \; xj} = 0$, $g=1$
\[
\begin{gathered}
  \mathcal{L}_u g = 2 \left( \frac{ \partial u }{ \partial y} \right) dy \otimes dx \\ 
  \mu\left( \frac{1}{2} \mathcal{L}_u g \right) \left( \frac{ \partial }{ \partial y} \right) = \mu \frac{ \partial u}{ \partial y } dx \overset{\sharp}{\mapsto } \mu \frac{ \partial u}{ \partial y} = \tau_x = \tau
\end{gathered}
\]



From Sec. 3.6.2., the geodesic equation of a time-dependent problem is given by Kambe's Eqn. (3.67), 
\[
\partial_tX - \text{ad}^*_XX = 0 
\]
(cf. Kambe (2009) \cite{TKambe2009})


\section{Momentum Theorems: Momentum flux}

\begin{itemize}
  \item Momentum theorems
  \item Newton's Second law, Newton's Force law
  \item Momentum flux
\end{itemize}

I'm following Chapter 4: Momentum Theorems of Sabersky, Acosta, Hauptmann, and Gates (1998)\cite{SAHG1998}.  

Let's imagine, in control volume $B=B(t)$ (which is, in general, dependent on time $t$), that we sum up all the momentum contributions at each point in $B(t)$, and then taking the total time derivative $\frac{d}{dt}$ of this sum $\Pi$ (which becomes an integral in this ``continuum''):
\begin{equation}\label{Eq:F=ma}
\begin{gathered}
  F = \dot{\Pi} \equiv \frac{d}{dt} \Pi = \frac{d}{dt} \int_{B(t)} \rho u \text{vol}^n \equiv \frac{d}{dt} ( \mathcal{L}_{\frac{ \partial }{ \partial t} + \mathbf{u} } m ) \mathbf{u} + m \mathcal{L}_{ \frac{ \partial }{ \partial t} + \mathbf{u} } \mathbf{u} = \\
  = 0 + \int_{B(t)} m \left( \frac{ \partial u}{ \partial t} + \mathcal{L}_{\mathbf{u}} \mathbf{u}  \right) = \int_{B(t)} m \left( \frac{ \partial u }{ \partial t} + u^k \frac{ \partial u}{ \partial x^k} \right)
\end{gathered}
\end{equation}
where in the second line, I used mass conservation, in that $\frac{d}{dt} \int m = \int_{B(t)} \left( \mathcal{L}_{ \frac{ \partial }{ \partial t} + \mathbf{u} } m \right) = 0$.  Also, note in the second line that $\mathcal{L}_{\mathbf{u}} \mathbf{u} = u^k \frac{ \partial \mathbf{u}}{ \partial x^k}$ for \emph{time-dependent} velocity vector field $\mathbf{u} = \mathbf{u}(t,x) \in \mathfrak{X}(\mathbb{R} \times N)$ by theorem, which I will recap here (because I (proudly) derived it on my own and haven't found it explicitly shown in any texts that I know of):  

\begin{theorem}
\begin{equation}
\mathcal{L}_{\widetilde{V}}W = \mathcal{L}_{ \frac{ \partial }{ \partial t} +V}W = \left( \mathcal{L}_{ \frac{ \partial}{ \partial t} V } W \right)_{(t,p)} = \left( \left( \frac{ \partial }{ \partial t} + V \right) W^j \right) \left. \frac{ \partial }{ \partial x^j} \right|_{(t,p)}
\end{equation}
for $\widetilde{V} = \frac{ \partial }{ \partial t} + \frac{ \partial }{ \partial u^1} \in \mathfrak{X}(\mathbb{R} \times M)$  and $W \in \mathfrak{X}(\mathbb{R} \times M)$
\end{theorem}

\begin{proof}
Let's define the Lie derivative:
\begin{equation}
\begin{gathered}
  \mathcal{L}_{\widetilde{V}}W = (\mathcal{L}_{\widetilde{V}}W)_{(t,p)} = \left. \frac{d}{ds} \right|_{s=0}(d\widetilde{\theta}_{-s})_{\widetilde{\theta}_s(t,p)}(W_{\widetilde{\theta}_s(t,p)})   = \lim_{s\to 0} \frac{ (d\widetilde{\theta}_{-s})_{\widetilde{\theta}_s(t,p)}( W_{\widetilde{\theta}_s(t,p)}) - W_{\widetilde{\theta}_s(t,p)} }{s}
\end{gathered}
\end{equation}


Use Case 1 of the proof of Lee's Theorem 9.38 \cite{JLee2012}, for showing $\mathcal{L}_VW = [V,W]$.  \\
Let open neighborhood $U \subseteq J \times M$, with $(t,p) \in U$.  On open $U$, choose smooth coordinates $(t,u^i)$ on $U$.  By Theorem 9.22 of Lee \cite{JLee2012}, that at a regular point $p\in M$, $\exists \, (u^i)$ coordinates s.t. $V_p = \frac{ \partial }{ \partial u^1}$, then consider 

\[
\widetilde{V} = \frac{ \partial }{ \partial t} + \frac{ \partial }{ \partial u^1} \in \mathfrak{X}(\mathbb{R} \times M)
\]
with $V(t)(p) = \frac{ \partial }{ \partial u^1} \in \mathfrak{X}(M)$.  (Remember, $V(t)$ is a vector-field that is time-dependent, but is on $M$.  I will use this as a justification for using Thm. 9.22).  

Now the flow $\widetilde{\theta}_s$ takes on these forms:
\[
\begin{gathered}
  \widetilde{\theta}^{(t,p)}(s) = \widetilde{\theta}(s,(t,p)) = \widetilde{\theta}_s(t,p) = \\
  = (\alpha(s,(t,p)) , \beta(s,(t,p))) = (s+t, \beta(s,(t,p)) )
\end{gathered}
\]
Given these conditions, that \\
$\beta(0,(t,p)) = p = (u^1,u^2, \dots u^n)$ and 
\[
\left. \frac{ \partial \beta}{ \partial s}(s, (t,p)) \right|_{s=0} = V(t,p) = \frac{ \partial }{ \partial u^1} = \left. \frac{d}{ds} \beta^{(t,p)}(s) \right|_{s=0}
\]
then a $\beta$ that satisfies these conditions above is 
\[
\beta(s,(t,p)) = \beta_s(t,p) = (u^1 + s, u^2 \dots u^n)
\]
so that we can conclude that 
\[
\widetilde{\theta}_s(t,p) = (t+s, u^1 + s, u^2 , \dots , u^n)
\]

For fixed $s$, then
\[
d(\widetilde{\theta}_{-s})_{\widetilde{\theta}_s(t,p)} =1_{T_{\widetilde{\theta}_s(t,p)}(\mathbb{R}\times M)}
\]
so that 
\[
\begin{gathered}
  d(\widetilde{\theta}_{-s})_{\widetilde{\theta}_s(t,p)}(W_{\widetilde{\theta}_s(t,p)}) = d(\widetilde{\theta}_{-s})_{\widetilde{\theta}_s(t,p)} \cdot W^j(t+s,u^1 +s, u^2 \dots u^n) \left. \frac{ \partial }{ \partial u^j} \right|_{\widetilde{\theta}_s(t,p)} = W^j(t+s,u^1+s, u^2 \dots u^n) \left. \frac{ \partial }{ \partial u^j} \right|_{(t,p)} \\
\Longrightarrow \left. \frac{d}{ds} \right|_{s=0} W^j(t+s,u^1+s, u^2 \dots u^n) \left. \frac{ \partial }{ \partial u^j} \right|_{(t,p)} = \left( \frac{\partial }{ \partial t} W^j(t,u^1 \dots u^n) + \frac{ \partial }{ \partial u^1 } W^j(t,u^1 \dots u^n) \right) \left. \frac{ \partial}{ \partial u^j} \right|_{(t,p)}
\end{gathered}
\]


Thus, we can conclude that 
\begin{equation}
  \boxed{ \mathcal{L}_{\widetilde{V}}W = \mathcal{L}_{ \frac{ \partial }{ \partial t} +V}W = \left( \mathcal{L}_{ \frac{ \partial}{ \partial t} V } W \right)_{(t,p)} = \left( \left( \frac{ \partial }{ \partial t} + V \right) W^j \right) \left. \frac{ \partial }{ \partial x^j} \right|_{(t,p)} }
\end{equation}
 

\end{proof}

In, Eq. \ref{Eq:F=ma}, I had used the fact that the Lie derivative should act like a derivative, similar to the product rule in calculus, on symmetric tensors, even on tensor products of different bundles: if one doubts the veracity of this, then explicitly, I calculate the following:
\[
\begin{gathered}
  \frac{ \partial }{ \partial t}(\rho u) \text{vol}^d + \mathcal{L}_u (\rho u\text{vol}^d) = \frac{ \partial \rho }{ \partial t} u \text{vol}^d + \rho \frac{ \partial u }{ \partial t} \text{vol}^d + d(\rho u i_u\text{vol}^d) + 0 = \frac{ \partial \rho }{ \partial t} u \text{vol}^d + \rho \frac{ \partial u}{ \partial t} \text{vol}^d + \text{div}(\rho u^j u) \text{vol}^d \otimes \frac{ \partial }{ \partial x^j} = \\
  = \frac{ \partial \rho }{ \partial t} u \text{vol}^d + \rho \frac{ \partial u }{ \partial t} \text{vol}^d + \text{div}(\rho u) \text{vol}^d \otimes u + \rho u^k \frac{ \partial u}{ \partial x^k} \text{vol}^d = \dot{m} \otimes u + m \left( \frac{ \partial u }{ \partial t} + u^k \frac{ \partial u }{ \partial x^k } \right)
\end{gathered}
\]
where $\dot{m} = \mathcal{L}_{\frac{ \partial }{ \partial t} + \mathbf{u}} m$ and $m := \rho \text{vol}^d$.  


Therefore, if there are no mass source terms inside $B(t)$, i.e. $\dot{m} =0 $ or $\frac{ \partial \rho}{ \partial t} + \text{div}(\rho u) = 0$, 
\begin{equation}\label{Eq:F=mafromLiederivative}
  \boxed{
    F = \int_B \dot{m} \otimes u + m \left( \frac{ \partial u}{ \partial t} + u^k \frac{ \partial u}{ \partial x^k} \right) = \int_B m \left( \frac{ \partial u}{ \partial t} + u^k \frac{ \partial u}{ \partial x^k} \right) 
} \text{ for } \dot{m}=0
\end{equation}



The \emph{exterior covariant derivative}, $D$, was defined in Frankel (2004) \cite{TFrankel2004}, Morita (2001) \cite{SMorita2001} (and of course can be searched for on wikipedia).  Recalling the definition of $D$:
\begin{equation}\label{Def:exteriorcovariantderivative}
\begin{aligned}
  &  D:\Omega^p(M;E)\to \Omega^{p+1}(M;E) \\ 
  &  D(\theta \otimes s) = d\theta \otimes s + (-1)^p \theta \wedge \nabla s
\end{aligned}
\end{equation}
where $\begin{aligned} & \quad \\
  & \theta \in \Omega^p(M) \\
  & s \in \Gamma(E) \end{aligned}$ and where $E \to M$ is a vector bundle.  

The situation we want to consider is where $M = \mathbb{R}\times N$, where $E=TN$, the tangent bundle, but on smooth manifold $N$, which represents only space.  For $\text{dim}N = d$, i.e. the dimension of $N$ is $d$, then we're considering
\[
D: \Omega^{d-1}(\mathbb{R} \times N ; TN) \to \Omega^d(\mathbb{R}\times N;TN)
\]  
with 
\[
u^k dS_k \otimes \rho \mathbf{u} = i_{\mathbf{u}} \text{vol}^d \otimes \rho \mathbf{u} \in \Omega^{d-1}(\mathbb{R} \times N; TN) = \Omega^{d-1}(\mathbb{R} \times N) \otimes TN
\]
Some clarification should deservedly be made here. We want our velocity vector field $\mathbf{u}$ to be \emph{time-dependent} and so if we can assume a (``natural'') foliation on the spacetime manifold $M$ into $\mathbb{R} \times N$, so that time $t\in \mathbb{R}$ in $\mathbb{R} \times N$, then $\mathbf{u}$ depend locally on coordinate values in $\mathbb{R}\times N$, i.e. $\mathbf{u} = \mathbf{u}(t,x^1,x^2, \dots x^d) \equiv \mathbf{u}(t,x)$.  

But, $\mathbf{u} \in TN$, i.e. $\mathbf{u}$ lives in $TN$, solely, because, since we have our particular foliation, there's no $\frac{ \partial }{ \partial t}$ ``time-vector'' component or ``part'' to $\mathbf{u}$, i.e. $\mathbf{u} = u^1 \frac{ \partial }{ \partial x^1} + u^2 \frac{ \partial }{ \partial x^2} + \dots + u^d \frac{ \partial }{ \partial x^d} = u^1(t,x) \frac{ \partial }{ \partial x^1} + u^2(t,x) \frac{ \partial }{ \partial x^2} + \dots + u^d(t,x) \frac{ \partial }{ \partial x^d}$.  There's no ``time part'' or ``time component'' i.e. $u^t$ for $u^t \frac{ \partial }{ \partial t}$.  So $\mathbf{u}$ ``lives soley on'' tangent bundle $TN$.    

Using the formula from the definition of the exterior covariant derivative $D$, Eq. \ref{Def:exteriorcovariantderivative}, 
\[
D(\theta \otimes s) = d\theta \otimes s + (-1)^p \theta \wedge \nabla s
\]
let us carefully calculate $D(i_{\mathbf{u}} \text{vol}^d \otimes \rho \mathbf{u})$ piece by piece.  Thus
\[
\begin{gathered}
  di_{\mathbf{u}} \text{vol}^d = d(u^k dS_k) = \text{div}\mathbf{u} \text{vol}^d
\end{gathered}
\]
\[
\begin{gathered}
  \mathbf{\nabla} (\rho \mathbf{u}) = \rho \nabla u + (\nabla \rho) u = \rho (du + Au) + (d\rho)\otimes u \\
  \Longrightarrow u^k dS_k \wedge \nabla (\rho u) = u^k dS_k \wedge (\rho (du+Au) + (d\rho)\otimes u
\end{gathered}
\]
Now
\[
\begin{gathered}
\epsilon_{kj_2 \dots j_d} dx^{j_2} \wedge \dots \wedge dx^{j_d} \wedge dx^l = \epsilon_{kj_2 \dots j_d} \epsilon^{j_2 \dots j_d l }_{ 12 \dots d} dx^1 \wedge \dots \wedge dx^d = (d-1)! \delta_k^l dx^1 \wedge \dots \wedge dx^d \text{ and } \\
du = \frac{ \partial u^m }{ \partial x^l} dx^l \otimes \frac{ \partial }{ \partial x^m}
\end{gathered}
\]
and so 
\begin{equation}\label{Eq:extcovderofmomentumfluxexplicit}
\begin{aligned}
  & u^k dS_k \wedge du = u^k \frac{ \partial u}{ \partial x^k} \text{vol}^n \\ 
  & u^k dS_k \wedge Au = u^l A^k_{ \; \; lj} u^j \text{vol}^d \otimes \frac{ \partial }{ \partial x^k} = \Gamma^k_{ \; \; ij } u^i u^j \text{vol}^d \otimes \frac{ \partial }{ \partial x^k} \\ 
  & u^k dS_k \wedge d\rho = u^k \frac{ \partial \rho }{ \partial x^k} \text{vol}^n 
\end{aligned}
\end{equation}
Taking a look at the last term and $\text{div}\mathbf{u}$ above, we can combine the 2 in such a matter:
\[
di_{\mathbf{u}} \text{vol}^d \otimes \rho \mathbf{u} + u^k \frac{ \partial \rho }{ \partial x^k} \text{vol}^d \otimes \mathbf{u} = \text{div}(\rho \mathbf{u}) \otimes \mathbf{u} \text{vol}^d
\]
For the other 2 terms left, of Eq. \ref{Eq:extcovderofmomentumfluxexplicit}, notice that 
\[
u^k dS_k \wedge ( \rho (du + Au)) = \rho (u^k \frac{ \partial u}{ \partial x^k} + \Gamma^k_{ \; \; ij} u^i u^j \frac{ \partial }{ \partial x^k}) \text{vol}^d = \rho ( \nabla_{\mathbf{u}} \mathbf{u}) \otimes \text{vol}^d
\]
so that we have something, $\nabla_{\mathbf{u}} \mathbf{u}$ that looks like part of a \emph{geodesic} equation.  

Thus, for $d$ odd (e.g. $d=3$), 
\begin{equation}\label{Eq:extcovder_boundarymomentumflux}
  \boxed{
    \begin{aligned}
      & D : \Omega^{d-1}(\mathbb{R} \times N, TN) \to \Omega^d(\mathbb{R}\times N,TN) \\ 
      & D(i_{\mathbf{u}} \text{vol}^d \otimes \rho \mathbf{u}) = D( u^k dS_k \otimes \rho \mathbf{u} ) = \rho(\nabla_{\mathbf{u}} \mathbf{u} ) \otimes \text{vol}^d + \text{div}(\rho \mathbf{u}) \text{vol}^d \otimes \mathbf{u}
\end{aligned}
}
\end{equation}

Suppose we can generalize Stoke's law for so-called vector-valued $p$-forms using $D$, i.e.
\[
\int_B D(\theta \otimes s) = \int_{\partial B} \theta \otimes s
\]


I can make this \emph{conjecture} or \emph{proposition}: If we start with defining the momentum flux on the surface boundary of the control volume $B=B(t)$, $\partial B$ and equate that to the force on the control volume:
\[
F = \frac{d}{dt} \Pi \equiv \dot{\Pi} = \int_{\partial B} u^j dS_j \otimes \rho u = \int_{\partial B} i_{\mathbf{u}}\text{vol}^d \otimes \rho \mathbf{u}
\]
where $\mathbf{u} \in \mathfrak{X}(\mathbb{R}\times N)$ and $u^jdS_j \otimes \rho \mathbf{u} \in \Omega^d(\mathbb{R} \times N; TN)$, then we obtain this expression involving an integral on the control volume $B(t)$, but using the ``generalized'' \emph{exterior covariant derivative} $D$, s.t. 
\begin{equation}\label{Eq:F=ma_extcovder}
  F = \dot{\Pi} = \int_{\partial B} i_{\mathbf{u}} \text{vol}^d \otimes \rho \mathbf{u} = \int_B D(i_{\mathbf{u}} \text{vol}^d\otimes \rho \mathbf{u}) = \int_B \rho (\nabla_{\mathbf{u}} \mathbf{u} ) \otimes \text{vol}^d + \text{div}(\rho \mathbf{u}) \text{vol}^d \otimes \mathbf{u}
\end{equation}


But how do we reconcile Eq. \ref{Eq:F=mafromLiederivative}, obtained from taking the total time derivative of an integral of a $d$-form on $\mathbb{R} \times N$, with this expression, Eq. \ref{Eq:extcovder_boundarymomentumflux}, derived using the exterior covariant derivative $D$?  Unfortunately, by comparing components in local coordinates, I don't have a good way of reconcile the two expressions, due to the term $\Gamma^k_{ \; \; ij} u^i u^j \frac{ \partial }{ \partial x^j}$ in $\nabla_{\mathbf{u}} \mathbf{u}$.  I really don't know after playing around extensively with the algebra; please let me know otherwise (see my email address in this document).  

Suppose we can do the usual Stoke's theorem on integrals of differential forms on only one ``part'' of a symmetric tensor, independent of the rest of the symmetric tensor, i.e.
\[
\int_V (d,-) \theta \otimes s \equiv \int_V d (\theta \otimes s) = \int_{ \partial V } \theta \otimes s
\]
where 
\[
\begin{aligned}
& (d,-) : \Omega^p(M) \otimes \Gamma(E) \to \Omega^{p+1}(M) \otimes \Gamma(E)  \\
  & (d,-) (\theta \otimes s) \equiv d (\theta \otimes s) = (d\theta) \otimes s
\end{aligned}
\]
on vector bundle $E\to M$.  

If this is the case, then
\[
\begin{gathered}
  (d,-)(\rho u^j u^k dS_k \otimes \frac{ \partial }{ \partial x^j} ) \equiv d(\rho u^j u^k dS_k) \otimes \frac{ \partial }{ \partial x^j} = \frac{ \partial ( \rho u^j u^k \sqrt{g} )}{ \partial x^l } dx^l \wedge \frac{ \epsilon_{kj_2 \dots j_d} }{(d-1)!} dx^{j_2} \wedge \dots \wedge dx^{j_d} \otimes \frac{ \partial }{ \partial x^j} = \frac{1}{\sqrt{g}} \frac{ \partial ( \rho u^j u^k \sqrt{g} )}{ \partial x^k} \text{vol}^d \otimes \frac{ \partial }{ \partial x^j} = \\
  = \left( \text{div}(\rho u) u^j + \rho u^k \frac{ \partial u^j }{ \partial x^k} \right) \text{vol}^d \otimes \frac{ \partial }{ \partial x^j } = \text{div}(\rho u) \text{vol}^d \otimes u + m \otimes \mathcal{L}_uu
\end{gathered}
\]

Thus, if this was the case, 
\[
\begin{gathered}
  F = \frac{d}{dt} \Pi \equiv \dot{\Pi} := \frac{d}{dt} \int_B m \otimes u = \int_B \dot{m} \otimes u + m \left( \frac{ \partial u}{ \partial t} + u^k \frac{ \partial u}{ \partial x^k} \right) = \int_B \frac{ \partial (\rho u)}{ \partial t} \text{vol}^d + \int_{ \partial B} \rho u^k dS_k \otimes u 
\end{gathered}
\]
Even if $\dot{m}=0$, i.e. there are no mass source terms inside control volume $B$, the same expression is obtained:
\[
\begin{gathered}
  F = \frac{d}{dt} \Pi \equiv \dot{\Pi} := \frac{d}{dt} \int_B m \otimes u = \int_B \dot{m} \otimes u + m \left( \frac{ \partial u}{ \partial t} + u^k \frac{ \partial u}{ \partial x^k} \right) = \int_B m\frac{ \partial  u}{ \partial t} + \int_B - \text{div}(\rho u) \text{vol}^d \otimes u  + \int_{ \partial B} \rho u^k dS_k \otimes u = \\
  =   \int_B \frac{ \partial ( \rho u )}{ \partial t} \text{vol}^d + \int_{ \partial B} \rho u^k dS_k \otimes u 
\end{gathered}
\]

Going back to Eq. \ref{Eq:F=ma_extcovder} involving exterior covariant derivative $D$, I can make the following adhoc addition to the $\int_{\partial B} {\mathbf{u}} \text{vol}^d \otimes \rho \mathbf{u} = \int_{\partial B} \rho u^k dS_k \otimes u $ term:
\[
\begin{gathered}
  F = \dot{\Pi} := \int_{\partial B} i_{\mathbf{u}} \text{vol}^d \otimes \rho \mathbf{u}  + \int_B \frac{ \partial (\rho u)}{ \partial t} \text{vol}^d =  \\
   = \int_B D(i_{\mathbf{u}} \text{vol}^d\otimes \rho \mathbf{u})  + \int_B \frac{ \partial (\rho u )}{ \partial t} \text{vol}^d = \int_B \rho (\nabla_{\mathbf{u}} \mathbf{u} ) \otimes \text{vol}^d + \text{div}(\rho \mathbf{u}) \text{vol}^d \otimes \mathbf{u} + \int_B \frac{ \partial ( \rho u )}{ \partial t} \text{vol}^d = \\
  = \int_B \rho(\nabla_{\mathbf{u}} \mathbf{u} ) \text{vol}^d + \dot{m} \otimes \mathbf{u} + m \otimes \frac{ \partial u }{ \partial t} = \boxed{ \int_B \dot{m} \otimes \mathbf{u} + m \otimes \left( \frac{ \partial \mathbf{u}}{ \partial t} + \nabla_{\mathbf{u}} \mathbf{u} \right) }
\end{gathered}
\]

Thus, depending upon whether we can use either exterior covariant derivative $D$  or $(d,-)$ for Stoke's law, then 
\begin{equation}\label{Eq:momentumtheorems}
\boxed{
\begin{gathered}
\text{ For } (d,-):  \Omega^{d-1}(N) \otimes TN \to \Omega^{d}(N) \otimes TN \\
\begin{gathered}
  F = \frac{d}{dt} \Pi \equiv \dot{\Pi} := \frac{d}{dt} \int_B m \otimes u = \int_B \dot{m} \otimes u + m \left( \frac{ \partial u}{ \partial t} + u^k \frac{ \partial u}{ \partial x^k} \right) = \int_B \frac{ \partial (\rho u)}{ \partial t} \text{vol}^d + \int_{ \partial B} \rho u^k dS_k \otimes u 
\end{gathered} \\
\text{ Otherwise, } \\
\text{ For } D:\Omega^{d-1}(N;TN)\to \Omega^{d}(N;TN) \\
\begin{gathered}
  F = \dot{\Pi} := \int_{\partial B} i_{\mathbf{u}} \text{vol}^d \otimes \rho \mathbf{u}  + \int_B \frac{ \partial (\rho u)}{ \partial t} \text{vol}^d =  \\
 =  \int_B \dot{m} \otimes \mathbf{u} + m \otimes \left( \frac{ \partial \mathbf{u}}{ \partial t} + \nabla_{\mathbf{u}} \mathbf{u} \right) 
\end{gathered}
\end{gathered}
}
\end{equation}
These 2 cases hold, in each case, whether or not mass is conserved, i.e. $\dot{m}=0$ or not, signifying whether there is a mass source term inside $B$, or not, respectively.  

My suspicion is that the second set of equations for the case of using $D$ is ``more true'' in that it holds for general Riemannian manifold $(N,g)$, with time-dependent velocity vector fields $\mathbf{u}$, in Eq. \ref{Eq:momentumtheorems}, because of the generality of the exterior covariant derivative $D$ and how $\nabla_{\mathbf{u}} \mathbf{u}$ becomes $u^k \frac{ \partial \mathbf{u}}{ \partial x^k}$, in the particular case of the Euclidean metric.  However, it was derived in an ad hoc manner.  Regardless, one aspect to investigate would be the \emph{de Rham} cohomology for $N$, to determine whether $(d,-)$ is valid or $D$ is valid.    

Take a look at Subsection 4.1 Introduction of Chapter 4 Momentum Theorems of Sabersky, Acosta, Hauptmann, and Gates (1998)\cite{SAHG1998}.  Consider Figure 4.1 and the steady fluid jet deflection by stationary channel.  Then
\[
\begin{gathered}
  F=\int_B \frac{ \partial ( \rho u)}{ \partial t} \text{vol}^d + \int_{\partial B} \rho u^j dS_j \otimes u = 0 + \int_{ (\partial B)_I} \rho u^j dS_j \otimes u + \int_{ (\partial B)_{III} } \rho u^j dS_j \otimes u = (-\rho_I u_I A_I) \mathbf{u}_I + \rho_{III} u_{III} A_{III} \mathbf{u}_{III}
\end{gathered}
\]
If velocity is constant, $u_I = u_{III}$, and density doesn't change, and $A_I = A_{III}$, the jet cross-sectional area, 
\[
\begin{aligned}
  & F = -\rho u A \mathbf{u}_I + \rho u A \mathbf{u}_{III} \\ 
  & F_x = \rho u^2 A (\cos{\theta}-1) \\ 
  & F_y = \rho u^2 A \sin{\theta}
\end{aligned}
\]
Keep in mind that this is the force on the fluid.  The force on the stationary channel is equal and opposite to this force, by Newton's 3rd. law.  
\[
|F| = \rho u^2 A ( \cos^2{\theta} - 2\cos{\theta} + 1 )^{1/2} = 2\rho u^2 A |\sin{ \frac{\theta}{2} } |
\]

\subsection{Differentiating Integrals of differential forms}

Frankel posed the question ``How does one compute the rate of change of an integral when the domain of integration is also changing?'' in Chapter 4 The Lie Derivative, Section 4.3 Differentiation of Integrals in Frankel (2004).  For time-dependent vector fields, he sketches the proof for Theorem 4.42 for a formula for ``bringing'' the total time derivative $\frac{d}{dt}$ ``into the integral'' of the differential form.  For a more explicit (and elucidating explanation in my opinion) derivation of this formula, I would suggest looking at Theorem 9.48, Fundamental Theorem on Time-Dependent Flows, on pp. 237 of Chapter 9 Integral Curves and Flows, in the subsection on Time-Dependent Vector Fields, of Lee (2012) \cite{JLee2012} and, in particular, the corresponding proof, for, the idea is that we'd want to flow along in $\mathbb{R} \times \mathbb{R} \times N$, in what Lee calls the ``flow domain'', and not just $\mathbb{R} \times N$, which is $N$, but parametrized by $s \in \mathbb{R}$; we'd want to parametrize, by $s\in \mathbb{R}$, $\mathbb{R} \times N$, instead.  

Taking from the proof of Theorem 9.48 of Lee (2012) \cite{JLee2012}, $\exists \, !$ smooth $\widetilde{\theta}: \widehat{\mathcal{D}} \to J \times N$, where interval $J \subset \mathbb{R}$, and $\widetilde{\mathcal{D}} \subset \mathbb{R} \times ( \mathbb{R} \times N)$, s.t. 
\[
\begin{gathered}
  \frac{ \partial \widetilde{\theta}}{ \partial s }(s,(t,x)) = (1, \frac{ \partial \beta}{ \partial s}(s,(t,x))) = \frac{ \partial }{ \partial t} + \mathbf{v}(s+t, \beta(s,(t,x)))  \text{ and } \\
  \beta(0,(t,x)) =x \qquad \, \text{ (i.e. initial condition) }
\end{gathered}
\]  
where $\beta : \widetilde{\mathcal{D}} \to N$ is smooth.  

Note that 
\[
\frac{ \partial \widetilde{\theta}}{ \partial s}(0,(t,x)) = \frac{ \partial }{ \partial t} + \mathbf{v}(t,0) \equiv \frac{ \partial }{ \partial t} + \mathbf{v}
\]
is the generator of the flow $\widetilde{\theta} = \widetilde{\theta}(s,(t,x))$, and we define 
\[
\begin{aligned}
& \widetilde{\theta}_s: \mathbb{R} \times N \to \mathbb{R} \times N \\ 
  & \widetilde{\theta}(s,(t,x)) =: \widetilde{\theta}_s(t,x)
\end{aligned}
\]
which we restrict onto appropriate open sets of $\mathbb{R} \times (\mathbb{R} \times N)$ (i.e. take heed that $\widetilde{\theta}_s$ is not valid on the entire $\mathbb{R} \times N$; see Lee (2012) \cite{JLee2012}).  

Thus, for $p$-form $\omega = \omega(t,x)$, a $p$-form on $\mathbb{R} \times N$, so that $\omega \in \Omega^p(\mathbb{R} \times N) \equiv \Omega^p(N)$ (I want to denote that the components of $\omega$ can depend on time, but also denote that there is no ``$\omega^t$'' time-component for $dt$ in $\omega$, and so I'll use both notations, interchangeably, and one discerns from context), then define 
\[
I(t) := \int_{ B(t)} \omega(t,x)
\]  
Thus, we can pull back onto the domain $B(t)$ from later times, say $s+t$, and so
\[
\begin{gathered}
  I'(t) \equiv \frac{d}{dt} I(t) := \lim_{s\to 0} \frac{ I(s+t) -I(t) }{ s} = \lim_{s\to 0} \frac{ \int_{B(s+t)} \omega(s+t,x) - \int_{B(t)} \omega(t,x) }{ s} = \lim_{s\to 0} \frac{ \int_{B(t) } \widetilde{\theta}^*_s\omega(t,x) - \int_{B(t)} \omega(t,x) }{s} = \\
   = \int_{B(t)} \lim_{s\to 0} \frac{ \widetilde{\theta}_s^* \omega(t,x) - \omega(t,x) }{ s }  =: \int_{B(t)} \mathcal{L}_{\frac{ \partial }{ \partial t } + \mathbf{v}(t,x) } \omega \equiv \int_{B(t)} \mathcal{L}_{\frac{ \partial }{\partial t} + \mathbf{v} } \omega
\end{gathered}
\]

Thus, without having to ``hand wave around'' the problem of dealing with a fixed control volume and then the volume occupied by the fixed mass that's changing with time, as in Section 4.2 General Derivation of the Momentum Theorem of Sabersky, Acosta, Hauptmann, and Gates (1998)\cite{SAHG1998}, we've shown here, with mathematical rigor, how to formally take the time derivative of integrals of differential forms.  

It appears that others have also worked on this mathematical formulation of Navier-Stoke's equations:

\url{https://pdfs.semanticscholar.org/182e/a7017fe4cd8c80bfac45c64cabd8e557903f.pdf}
A Nonlinear Analysis of the Averaged Euler Equations.  Holm, Kouranbaeva, Marsden, Ratiu, Shkoller.  

\url{https://www.math.ucdavis.edu/~shkoller/jdg.pdf}  Shkoller.  ANALYSIS ON GROUPS OF DIFFEOMORPHISMS
OF MANIFOLDS WITH BOUNDARY AND THE AVERAGED MOTION OF A FLUID.  





\subsection{Angular Momentum}

7.15.6 Noether's theorem for rotations.

From rotational invariance, consider \\
Global rotation 
\[
\text{ Global rotation } \qquad \, v(x) \mapsto v'(x) = Rv(x), \qquad \, R\in SO(3), \quad \, \forall \, x \in N
\]
(equivalent to uniform rotation)

Take rotation vector of rotation angle $|\delta \widehat{\theta}|$, about axis $\mathbf{e}$, 
\[
\delta \widehat{\theta} = | \delta \widehat{\theta} | \mathbf{e}
\]

Then,  \\
infinitesimal global transformation defined, as, for \\
\phantom{ \qquad \,  } position vector $\mathbf{x}$ 
\[
\mathbf{x}' = \mathbf{x}+ \delta \widehat{\theta} \times \mathbf{x}
\]

From Eq. (7.147) of Kambe (2009) \cite{TKambe2009}, 
\[
\mathcal{L}_T = \mathcal{L}_F + \mathcal{L}_{\epsilon} + \mathcal{L}_B = \frac{1}{2} \int_N \langle v, v\rangle \rho \text{vol}^d - \int_N \epsilon(\rho,s) \rho \text{vol}^d + \mathcal{L}_B
\]
$\mathcal{L}_B$ depends on gauge potential $A$, function of $a$ only, and $a=a(x,t)$.  
\[
\delta \underline{I} = \left[ \int_N \langle v, \xi \rangle \rho \text{vol}^d \right]_{t_0}^t + \int_{t_0}^{t_1} dt \oint_S p \langle \eta, \xi \rangle dS - \int_{t_0}^{t_1} dt \int_N \langle (\nabla_t v + \rho^{-1} \text{grad} p ), \xi \rangle \rho \text{vol}^d
\]
\[
\delta L_T = \frac{ \partial }{ \partial t} \int_N \langle v, \xi \rangle \rho \text{vol}^d + \oint_S p \langle \xi , \eta \rangle dS - \int_N \langle (\nabla_t v + \frac{1}{\rho } \text{grad}p ), \xi \rangle \rho \text{vol}^d
\]
where $\xi = \delta \widehat{\theta} \times x$

Now
\[
\nabla_t  v+ \frac{1}{\rho } \text{grad}p = 0 
\]
since it's equation of motion of an ideal gas.  

From vector identities:
\[
\langle v, \delta \widehat{\theta} \times x \rangle = \langle \delta \widehat{\theta}, x \times v \rangle \equiv \delta \widehat{\theta} \cdot (x\times v)
\]
So
\[
\begin{gathered}
  \Longrightarrow \delta \widehat{\theta} \cdot \frac{ \partial }{ \partial t} \int_N x \times v \rho \text{vol}^d =: \delta \widehat{\theta} \cdot \frac{ \partial }{ \partial t} \mathbf{L}(N) \text{ where } \\ 
   \delta \widehat{\theta} \cdot \int_S x \times p \mathbf{n} dS = - \delta \widehat{\theta} \cdot \mathbf{N}(S)
\end{gathered}
\]

$\mathbf{N}(S)$ is resultant moment of pressure force $-p\mathbf{n}dS$ acting on surface element $dS$ from outside
\[
\Longrightarrow \frac{ \partial }{ \partial t} \mathbf{L}(M) = \mathbf{N}(S)
\]
Thus from Noether's theorem, total angular momentum conservation law, from $SO(3)$ gauge invariance.  



\section{Compressible Fluids-One-Dimensional Flow}

\subsection{Thermodynamic Preliminaries}

For a \textbf{fixed mass} system undergoing an incremental process:

\begin{equation}
dQ + dW = dE \qquad \, (\text{Sabersky, et.al., Eqn. 9.1})
\end{equation}
or rather
\begin{equation}
\begin{gathered} 
Q - W = dU \text{ while promoting $U$: } \\
U \mapsto E := U + \text{K.E.} + \text{P.E.} 
\end{gathered}
\end{equation}
where $U \equiv $ internal energy.

When K.E. and P.E. negligible, and for a reversible process, 
\begin{equation}
W = pdV
\end{equation} 	
with volume $V$.

If $\dot{m} = 0$, given $\rho V = m$ or $V = m / \rho$, 
\[
W = pdV = pd(m/\rho ) \text{ or } W / m := w = p d\left( \frac{1}{\rho } \right) 
\]
so that 
\[
Q -W = dU \text{ or } du = q - p d \left(\frac{1}{\rho} \right) 
\]

For a \emph{fixed control volume}, Sabersky, et. al. \cite{SAHG1998} mentions that Eqn. 9.2 is derived later (in the Energy Equation section, Sec. 9.4). 

Going back to the \textbf{fixed mass, reversible process} case, now $Q = \tau d\sigma$. 

Let $q := \frac{Q}{m} = \tau d\left( \frac{\sigma}{m} \right) =: \tau ds$ where $s := \frac{\sigma}{m}$

\begin{equation}
\begin{gathered}
\tau ds = du + p d\left( \frac{1}{\rho} \right) \\
H = U + pV \Longrightarrow dH = dU + Vdp + pdV \xrightarrow{\frac{1}{m}} dh = du + \frac{1}{\rho} dh + pd\left( \frac{1}{\rho} \right) \\
\tau ds = dh - \frac{1}{\rho} dp
\end{gathered}
\end{equation}

Experimentally, for many gases, change in internal energy is related to heat capacity:
\[
\begin{aligned} 
& Q - pdV = dU \\ 
& Q = Q(\tau, V) =  \left( \frac{ \partial Q}{ \partial \tau} \right)_V d\tau + \left( \frac{ \partial Q}{ \partial V} \right)_{\tau} dV 
\end{aligned} 
\]
Curve $c \subset \Sigma$ s.t. $V$ constant over $c \Longrightarrow$ 
\[
\begin{gathered}
Q(\dot{c}) =  \left( \frac{ \partial Q}{ \partial \tau } \right)_V d\tau (\dot{c}) + 0 \Longrightarrow dU(\dot{c}) = C_V d\tau (\dot{c}) \text{ where } C_V := \left( \frac{ \partial Q}{ \partial \tau} \right)_V
\end{gathered}
\]

$H = U + pV$ so 
\[
dH = dU + Vdp + pdV = \tau d\sigma +Vdp
\]

For $\sigma = \sigma(\tau, p)$ and curve $\gamma$ where $p$ constant $\Longrightarrow $,
\[
\begin{gathered}
	dH(\dot{\gamma}) = \tau d\sigma(\dot{\gamma}) + 0 = \tau d\sigma(\dot{\gamma}) \Longrightarrow dH(\dot{\gamma}) = \tau \left( \frac{ \partial \sigma}{ \partial \tau} \right)_p d\tau(\dot{\gamma} ) \xrightarrow{\frac{1}{m} } \\ 
	\xrightarrow{ \frac{1}{m}} dh = \tau \left( \frac{ \partial s }{ \partial \tau} \right)_p d\tau := c_p d\tau, \text{ where } c_p = \tau \left( \frac{ \partial s}{ \partial \tau} \right)_p \qquad \, (\text{fixed mass case}) 
\end{gathered}
\]

Assume specific heats are constant, then 
\[
\begin{aligned}
& h = c_p (\tau - \tau_r) + h_r \\ 
& u = c_V (\tau - \tau_r) + u_r
\end{aligned}
\]
where subscript $r$ denotes some reference state (cf. Sabersky, et. al. \cite{SAHG1998}).

Given $pV = N\tau \xrightarrow{ (m_0 / m_0) } p = \rho \tau / m_0 \xrightarrow{d} dp = \frac{ \tau d\rho + \rho d\tau }{m_0} $.

\[
\begin{gathered}
dh = c_p d\tau = du + \frac{dp}{ \rho } + p d\left(\frac{1}{\rho} \right) = c_V d\tau + \frac{dp}{\rho } + \frac{-p}{\rho^2} d\rho = c_V d\tau + \frac{p}{\rho^2 } d\rho + \frac{d\tau }{m_0} - \frac{p}{\rho^2 } d\rho = \\
c_V d\tau + \frac{ d\tau}{m_0} \\
\Longrightarrow c_p - c_V = \frac{1}{m_0}
\end{gathered}
\]
or
\begin{equation}
\begin{gathered}
dH = C_p d\tau = dU + Vdp + pdV = C_V d\tau + V (d\left(\frac{N\tau}{V}\right) ) + p dV = \\
= C_V d\tau + Nd\tau + \frac{N}{-V} \tau dV + pdV = C_V d\tau + Nd\tau \text{ or }  \\ 
\boxed{ C_p - C_V = N}
\end{gathered}
\end{equation}
Compare this with Eqn. 9.6 in Sabersky, et. al. \cite{SAHG1998}.

Consider (empirically verified) perfect (ideal) gas law $pV = N\tau$. 

For isentropic (adiabatic) process, 
\[
Q = \tau d\sigma = 0 = dU + W = C_V d\tau + pdV = C_V d\tau + \frac{N\tau}{V} dV 
\]
Consider any path \\
$\begin{aligned} 
& \gamma : \mathbb{R} \to \Sigma \\
& \gamma(t) = (\tau(t), V(t)) \end{aligned}$ \qquad \, in $\Sigma$ so \\
$\dot{\gamma}(t) = \dot{\tau} \frac{ \partial }{ \partial \tau } + \dot{V} \frac{ \partial }{ \partial V} \in \mathfrak{X}(\Sigma)$.

Thus, $0 = C_V \dot{\tau} + \frac{N\tau}{V} \dot{V}$ or $\frac{ \dot{\tau}}{\tau} + \frac{N}{C_V}\frac{\dot{V}}{V} = 0$.
\[
\xrightarrow{\int dt} \ln{ \frac{ \tau_f}{\tau_i} } + \frac{N}{C_V} \ln{ \frac{V_f}{V_i} }=  0 \text{ or } \frac{\tau_f}{\tau_i} = \left( \frac{V_i}{V_f} \right)^{\gamma - 1} = \frac{\tau_f}{\tau_i} = \left( \frac{\rho_f}{\rho_i} \right)^{\gamma - 1}
\]
where $\frac{N}{C_V} = \frac{C_p - C_V}{C_V} = \gamma -1$.

Now $pV = N\tau = (C_p - C_V)\tau$ or $V= \frac{ (C_p - C_V) \tau}{p}$
\[
\begin{gathered}
\frac{V_i}{V_f} = \left( \frac{\tau_i}{\tau_f} \right) \left( \frac{p_f}{p_i} \right) \\
\frac{\tau_f}{\tau_i} = \left( \frac{\tau_i}{\tau_f} \right)^{\gamma -1} \left( \frac{p_f}{p_i} \right)^{\gamma -1}
\end{gathered} 
\]
Doing $1 + (\gamma - 1)$ in the exponent, we get
\begin{equation}
\Longrightarrow \boxed{ \frac{\tau_f}{\tau_i} = \left( \frac{p_f}{p_i} \right)^{\frac{\gamma -1 }{\gamma } } }
\end{equation}

\subsection{Speed of Sound}

cf. Sec. 9.3 "The Speed of Sound", Sabersky, et. al. \cite{SAHG1998}.

The physiological effects of speech and hearing are related to the transmission and detection of tiny pressure disturbances. \\
Examine how infinitesimal pressure or density distributions propagate through compressible fluid. 

\quad \\ 
Consider piston-tube arrangement: long, thermally insulated tube, filled with compressible fluid, and moveable, close-fitting, frictionless piston. \\
Piston moves very slowly (toward right) at $\Delta V \Longrightarrow$ fluid compressed in front of piston $\Longrightarrow p + \delta p , \rho + \delta rho$ \\
This change will move along tube at steady speed $V_s$. \\
Behind this advancing front (left of) fluid properties constant as long as piston moves steadily. 

\quad \\ 
\textbf{Because of continuity}, \emph{fluid behind advancing front moves same speed as piston}, \\
while fluid in front of advancing pressure pulse (right of) still stationary.

\quad \\
If small pressure disturbances, frictionless movement, no heat flow $\Longrightarrow $ reversible, isentropic process. \\
In fixed reference frame, flow unsteady. \\
Consider boosting to moving reference frame at speed $V_s$.  

\quad \\ 
Lab frame: $p + \delta p, \rho + \delta \rho, u = \Delta V$ (left of advancing front); $p, \rho, u = 0$ (right of advancing front) \\
$+V_s$ frame: $p+\delta p, \rho + \delta \rho, u' = \Delta V- V_s$; $p, \rho, u' = -V_s$

\quad \\ 
Neglect viscous effects. Neglect elevation changes.

Now the key is to, in this case (analysis), use the \emph{surface area} form of the fluid flow equations.

mass conservation:
\[
\begin{gathered}
\frac{d}{dt} m = 0 = \frac{d}{dt} \int_{V_0} \rho \text{vol}^n = \int_{V_0} \frac{ \partial \rho }{ \partial t} \text{vol}^n + \int_{\partial V_0} i_{\mathbf{u}} \rho \text{vol}^n = 0 + \int_{\partial V_0} \rho u \cdot dS \\ 
\Longrightarrow (\rho + \delta \rho) (\Delta V - V_s) = \rho (-V_s) \text{ or } \rho V_s = (\rho + \delta \rho) (V_s - \Delta V) \text{ or } \rho \Delta V = \delta \rho V_s
\end{gathered}
\]

Newton's 2nd. law:
\[
\begin{gathered}
F = \frac{d}{dt} \Pi = \int_B \frac{ \partial ( \rho u)}{ \partial t} \text{vol}^d + \int_{\partial B} \rho u^k dS_k \otimes u = 0 + \int_{\partial B} \rho u^k dS_k \otimes u = \int_{\partial B} T^i_{ \, \, j } dS^j = \int_{\partial B} - p g^i_{\, \, j} dS^j   \\
\Longrightarrow \begin{gathered}
\rho V_s^2 - (\rho + \delta \rho) (\Delta V - V_s)^2 = - ( p - (p + \delta p)) = \delta p \\ 
\rho V_s^2 - \rho V_s (V_s - \Delta V) = \rho V_s \Delta V = \delta p  \\
\Longrightarrow \frac{ \delta p }{ \delta \rho } = \frac{ \rho V_s \Delta V}{ \frac{ \rho \Delta V}{ V_s} } \text{ or } V_s^2 = \frac{ \delta p}{ \delta \rho} 
\end{gathered}
\end{gathered}
\]
with $T = -pg$ for a perfect fluid ($T$ is the stress tensor).

Assume compression occur reversibly and adiabatically, so entropy $\sigma$ constant. 
\[
p = p(\rho) \Longrightarrow \delta p = \left( \frac{ \partial p}{\partial \rho} \right)_s \delta \rho
\]
So $V_s^2 = \left( \frac{ \partial p }{ \partial \rho} \right)_s$

EY : 20190820 Check this: $\left( \frac{\partial p }{\partial \rho} \right)_s = \gamma \left( \frac{ \partial p }{\partial \rho} \right)_{\tau}$ so for perfect gas, $a^2 = \gamma \tau$



\subsection{The Energy Equation}

Since 
\[
h_0 = h_1 + \frac{u_1^2}{2}
\]
then
\[
\frac{C_p}{MN} \tau_0 = \frac{C_p \tau_1}{MN} + \frac{u_1^2}{2} = \frac{C_p \tau_2}{MN} + \frac{u_2^2}{2} \text{ or } \tau_0 = \tau + \frac{u^2}{ 2 \frac{C_p}{MN} }
\]
Now the speed of sound at a particular point along the flow, $1$, is 
\[
a_1 = \sqrt{ \gamma RT_1 } = \sqrt{ \gamma \frac{\tau_1}{M} }
\]
and so 
\[
\Longrightarrow \tau_0 = \tau + \frac{ \mathfrak{M}^2 \gamma \tau }{ 2 \frac{C_p}{M}} = \tau \left( 1 + \frac{ \mathfrak{M}^2 (\gamma-1) }{2} \right)
\]
If $\mathfrak{M}=1$, when $\tau_1=\tau^*$, 
\[
\tau_0 = \tau^* \left( 1 + \frac{\gamma -1}{2} \right) = \tau^* \left( \frac{\gamma + 1 }{2} \right)
\text{ or } \tau^* = \frac{ 2\tau_0 }{ \gamma + 1 } 
\]

\begin{equation}\label{Eq:criticalvelocity}
a^* = \sqrt{ \gamma R T^*} = \sqrt{ \frac{\gamma \tau^* }{M} } = \sqrt{ \frac{2\gamma \tau_0 }{ M(\gamma + 1 ) } } = \left( \frac{2\gamma R T_0 }{\gamma + 1 } \right)^{1/2} \Longrightarrow u^* \equiv a^*  = \left( \frac{2\gamma RT_0 }{ \gamma + 1 } \right)^{1/2} = \sqrt{ \frac{2\gamma \tau_0}{M(\gamma +1) } }
\end{equation}

If $u > u^*$, 
\[
\begin{gathered}
  \mathfrak{M} := \frac{u}{a} = \frac{u}{ \sqrt{ \frac{ \gamma \tau }{ M } } } > \sqrt{ \frac{ 2\tau_0}{ \tau (\gamma + 1 ) } } = \sqrt{ \frac{2}{\gamma +1 } (1 + \frac{ \mathfrak{M}^2(\gamma - 1) }{2}  ) } \\ 
  \Longrightarrow \mathfrak{M}^2 > \frac{2}{\gamma +1 } + \mathfrak{M}^2 \frac{ (\gamma - 1) }{ \gamma + 1 } \text{ or } \mathfrak{M}^2 \left( \frac{2}{\gamma + 1} \right) > \frac{2}{\gamma +1 } \\
  \Longrightarrow \mathfrak{M} > 1
\end{gathered}
\]
So if $u > u^*$, then $\mathfrak{M}>1$.  Likewise, \\
\phantom{So } if $u < u^*$, then $\mathfrak{M}<1$

Thus, the name $u^*\equiv a^*$ \emph{critical velocity}

\subsection{Normal Shock Waves}

Use moving reference frame in which shock is stationary, and resulting steady.




Continuity:
\[
\rho_1 v_1 A = \rho_2 v_2 A \Longrightarrow \rho_1 v_1 = \rho_2 v_2 
\]

momentum equation: using
\[
\Pi^{ij} = \rho u^i u^j + p g^{ij}
\]
then
\[
\rho_1 v_1^2 + p_1 = \rho_2 v_2^2 + p_2 
\]

energy equation:
\[
h_0 = h_1 + \frac{v_1^2}{2} = h_2 + \frac{v_2^2}{2} \text{ or } \frac{C_p \tau_1}{MN} + \frac{v_1^2}{2} = \frac{C_p\tau_2}{MN} + \frac{v_2^2}{2} \Longrightarrow \frac{C_p R T_1}{N} + \frac{v_1^2}{2} = \frac{C_pR}{N} T_0
\]

Also, note that I end up using this heat capacity \emph{ for the ideal gas } relation all the time in (rocket) propulsion:
\[
C_p = \gamma C_v = \frac{\gamma N}{\gamma -1}  \text{ since } C_p = C_v + N \text{ or } \gamma = 1 + N/C_V
\]

With mass continuity, momentum conservation, and the energy equation (Bernoulli invariant), then, using Python's sympy to do the algebra, detailed in file \verb|fluid.py|
\begin{lstlisting}
# mass conservation
massconsEq = Eq(rho_1*u_1,rho_2*u_2)

# momentum flux conservation
momconsEq  = Eq(rho_1*u_1**2+p_1,rho_2*u_2**2+p_2)

# energy equation or Bernoulli invariant
Bernoulli_invariant_1to2_Eq = Eq( C_p*R*T_1/N + u_1**2/2, C_p*R*T_2/N + u_2**2/2)

# stagnation enthalpy relation
stagh1Eq = Eq( C_p*R*T_0/N , C_p*R*T_1/N + u_1**2/2)
stagh2Eq = Eq( C_p*R*T_0/N , C_p*R*T_2/N + u_2**2/2)

# ideal gas law
ideal_gas1Eq = Eq(p_1,rho_1*R*T_1)
ideal_gas2Eq = Eq(p_2,rho_2*R*T_2)

# This reproduces Eq. (9.26) of Sabersky, Acosta, Hauptmann, Gates pp. 357, Sec. 9.6., Normal Shock Waves !!!
(momconsEq.subs(p_1,ideal_gas1Eq.rhs).subs(p_2,ideal_gas2Eq.rhs).subs(rho_2,solve(massconsEq,rho_2)[0])/rho_1).simplify()
# R*T_1 - R*T_2*u_1/u_2 + u_1**2 - u_1*u_2

# This reproduces Eq. (9.27) of Sabersky, Acosta, Hauptmann, Gates pp. 357, Sec. 9.6., Normal Shock
 Waves, where stagnation temperature relation was substituted into momentum and continuity equation
PrandtlEq1d = momconsEq.subs(p_1,ideal_gas1Eq.rhs).subs(p_2,ideal_gas2Eq.rhs).
subs(rho_2,solve(massconsEq,rho_2)[0]).subs(T_1,solve(stagh1Eq,T_1)[0]).subs(T_2,solve(stagh2Eq,T_2)[0])

PrandtlEq1d = PrandtlEq1d.subs(C_p, gamma*N/(gamma-1) )

solve(PrandtlEq1d,u_1**2)[0]
# (2*R*T_0*gamma*u_1 - 2*R*T_0*gamma*u_2 + gamma*u_1*u_2**2 + u_1*u_2**2)/(u_2*(gamma + 1))
# Thus, writing this out on paper, we get the desired result, Prandtl's equation (9.28) on pp. 357 of 
# Sabersky, Acosta, Hauptmann, Gates
\end{lstlisting}
and thus
\[
u_1u_2 = \frac{2RT_0\gamma}{\gamma +1} = (a^*)^2
\]
for, the critical velocity was derived from only the energy equation (or Bernoulli invariant) and Mach \emph{definition}, and was shown, explicitly, that if the velocity $u$ at a point is greater than this critical velocity $u^*\equiv a^*$, then the flow is supersonic (cf. Eq. \ref{Eq:criticalvelocity}):
\[
u^* \equiv a^*  = \left( \frac{2\gamma RT_0 }{ \gamma + 1 } \right)^{1/2} = \sqrt{ \frac{2\gamma \tau_0}{M(\gamma +1) } }
\]
So if $u_1 > a^*$, then $u_2 < a^*$, and so approaching flow is supersonic and downstream flow is subsonic.   \\
Second law of thermodynamics forbids the other way.

Note that 
\[
\begin{gathered}
  u_1 u_2 = \frac{2RT_0 \gamma }{ \gamma +1} \text{ or } \mathfrak{M}_1\mathfrak{M}_2 = \frac{2}{\gamma +1} \frac{T_0 }{\sqrt{ T_1 T_2 }}
\end{gathered}
\]

(9.16) from Sabersky,
\begin{equation}
  \boxed{ \mathfrak{M}_2^2 = \frac{ \mathfrak{M}_1^2 (\gamma -1) + 2 }{ 2\gamma \mathfrak{M}_1^2 - \gamma + 1 } }
\end{equation}

Although preceding results obtained for constant-area duct, they're valid in gradually varying duct \\
\phantom{Although} shock region very thin; order of a few molecular mean free path lengths; area change across the shock is then usually negligible for practical purposes

Preceding, viscosity effects near wall ignored; since velocity must still be $0$ at wall, supersonic flow must revert to subsonic in near-wall region.\cite{SAHG1998}



Now recall that for $U=U(\tau,V)$ (in general, $U=U(\tau)$ for perfect ideal gas), 
\[
\tau d\sigma = dU + pdV = \left( \frac{ \partial U}{ \partial \tau} \right)_V d\tau + \left( \frac{ \partial U}{ \partial V} \right)_{\tau} dV + pdV = C_V d\tau + \left( \left( \frac{ \partial U}{ \partial V} \right)_{\tau} + p \right) dV
\]
If $U=U(\tau)$, 
\[
\begin{gathered}
  \tau d\sigma = C_V d\tau + p dV \text{ or } d\sigma = \frac{C_V}{\tau} d\tau + \frac{p}{\tau} dV = \frac{C_V}{\tau} d\tau + \frac{N}{V} dV  \\
  \Longrightarrow \int_{\gamma} d\sigma = \sigma_2 - \sigma_1 = C_V \ln{ \left( \frac{\tau_2}{\tau_1} \right) } + N \ln{\frac{ V_2}{V_1}} = C_V \ln{ \left( \frac{\tau_2}{\tau_1} \right) } + N\ln{ \left( \frac{\rho_1}{\rho_2} \right) }
\end{gathered}
\]


From momentum conservation, and ideal gas law and Mach definition,
\[
\begin{gathered}
  \rho_1 u_1^2 + p_1 = \rho_2 u_2^2 + p_2 \text{ or } p_1 - p_2 = \rho_2u_2^2 - \rho_1u_1^2 = \frac{p_2}{RT_2} u_2^2 - \frac{p_1}{RT_1} u_1^2 = \gamma \left( p_2 \mathfrak{M}_2^2 - p_1\mathfrak{M}_1^2 \right) \text{ or } \frac{p_1}{p_2} = \frac{1+\gamma \mathfrak{M}_2^2}{ 1 + \gamma \mathfrak{M}_1^2 }
\end{gathered}
\]
For real shocks, $\mathfrak{M}_1 > \mathfrak{M}_2$, so $p_1 < p_2$

Now
\[
\begin{gathered}
  \frac{\rho_1}{\rho_2} = \frac{p_1T_2}{p_2T_1} \\ 
  \frac{T_1}{T_2} = \frac{ 1 + \frac{\gamma-1}{2} \mathfrak{M}_2^2 }{ 1 + \frac{\gamma-1}{2} \mathfrak{M}_1^2 } \\
  \frac{\rho_1}{\rho_2} = \frac{(1+\gamma \mathfrak{M}_2^2 )}{ (1 + \gamma \mathfrak{M}_1^2 ) } \frac{ (1 + \frac{\gamma -1}{2} \mathfrak{M}_2^2 ) }{ (1 + \frac{\gamma-1}{2} \mathfrak{M}_1^2)}
\end{gathered}
\]


Shock waves are highly irreversible, since very large velocity and temperature gradients occur through shock itself; hence frictional, or dissipative effects must be present.\cite{SAHG1998}


\[
\sigma_2 - \sigma_1 = \frac{N}{\gamma -1} \ln{ \left[ \frac{1 + \frac{\gamma -1}{2} \mathfrak{M}_1^2 }{ 1 + \frac{\gamma-1}{2} \mathfrak{M}_2^2 } \right] } + N \ln{ \left[ \frac{1 + \gamma \mathfrak{M}_2^2 }{ 1 + \gamma \mathfrak{M}_1^2 } \frac{ ( 1 + \frac{\gamma -1}{2} \mathfrak{M}_2^2 ) }{ (1 + \frac{\gamma-1}{2} \mathfrak{M}_1^2) } \right] }
\]
From Sabersky (9.29),(9.16),(9.31) \cite{SAHG1998}
\[
\Longrightarrow s_2 - s_1 = \frac{R\gamma}{\gamma -1} \ln{ \left[ \frac{2}{(\gamma +1) \mathfrak{M}_1^2} + \frac{\gamma -1}{\gamma +1} \right] } + \frac{R}{\gamma -1} \ln{ \left[ \frac{2\gamma \mathfrak{M}_1^2 }{ \gamma +1 } - \frac{\gamma -1}{\gamma +1 } \right] }
\]



\part{Fluid Mechanics (revisited)}

\section{Conservation}

\begin{equation}
  \frac{ \partial \rho }{ \partial t } + \nabla \cdot \mathbf{j} = 0 
\end{equation}

\[
\begin{gathered}
  \frac{d}{dt} m \equiv \dot{m} = \frac{d}{dt} \int \rho \text{vol}^n = \int \left( \frac{ \partial \rho}{ \partial t} + \mathcal{L}_u \rho \right)\text{vol}^n = \int \left( \frac{ \partial \rho }{ \partial t} + (\mathbf{d}i_u + i_u \mathbf{d})\rho \right)\text{vol}^n = \int \frac{ \partial \rho }{ \partial t} + \mathbf{d}i_u \rho \text{vol}^n 
\end{gathered}
\]
Now
\[
\begin{gathered}
  \text{vol}^n = \frac{ \sqrt{g}}{n!} \epsilon_{i_1 \dots i_n} dx^{i_1} \wedge \dots \wedge dx^{i_n} \\ 
  i_{\mathbf{u}} \text{vol}^n = \frac{ \sqrt{g}}{(n-1)!} \epsilon_{i_1 \dots i_n} u^{i_1} dx^{i_2} \wedge \dots \wedge dx^{i_n} \\
  \xrightarrow{ \mathbf{d}} \frac{ \epsilon_{i_1 \dots i_n}}{ (n-1)!} \frac{ \partial (\sqrt{g} u^{i_1} \rho ) }{ \partial x^j} dx^j \wedge dx^{i_2} \wedge \dots \wedge dx^{i_n} = \frac{1}{\sqrt{g}} \frac{ \partial ( \sqrt{g} u^j \rho ) }{ \partial x^j} \text{vol}^n 
\end{gathered}
\]

\[
\Longrightarrow \frac{ \partial \rho }{ \partial t} + \text{div} j = 0 \text{ or } \frac{ \partial \rho}{ \partial t} + \frac{1}{\sqrt{g}} \frac{ \partial ( \sqrt{g} u^j \rho ) }{ \partial x^j} = 0 
\]

As a sanity check, consider a change of coordinates from cylindrical to Cartesian coordinates.  

Consider $g=  g_{ij} dx^i \otimes dx^j \in T^*M \otimes T^*M \equiv \otimes^2 T^*M$.  

For smooth (embedding or diffeomorphism) $F : N\to M$, \\
in our particular case, $F(r,\phi,z) = (x,y,z) = \left( \begin{matrix} r\cos{\phi } \\ r\sin{\phi} \\ z \end{matrix} \right)$

Now the pullback is $F^*g \in \otimes^2T^*N$
\[
\begin{gathered}
  F^* g(X,Y) = g(F_*X, F_*Y) = g\left( \frac{ \partial y^j}{ \partial x^i }X^i \frac{ \partial }{ \partial y^j}, \frac{ \partial y^k}{ \partial x^l} Y^l \frac{ \partial }{ \partial y^k} \right) = \frac{ \partial y^j}{ \partial x^i} \frac{ \partial y^k}{\partial x^l} X^i Y^l g\left( \frac{ \partial }{ \partial y^j}, \frac{ \partial }{ \partial y^k } \right) = \frac{ \partial y^j}{ \partial x^i} \frac{ \partial y^k}{ \partial x^l} X^i Y^l g_{jk} \\
\Longrightarrow  (F^*g)_{ij} = \frac{ \partial y^l}{ \partial x^i} \frac{ \partial y^m}{ \partial x^j} g_{lm}
\end{gathered}
\]
If $g_{jk} = \delta_{jk}$ (usual Euclidean metric),
\[
(F^*g)_{ij} = \frac{ \partial y^k}{ \partial x^i } \frac{ \partial y^k}{ \partial x^j} = \left( \frac{ \partial y^i}{ \partial x^k}\right)^T \frac{ \partial y^k}{ \partial x^j} = (D_xy)^T D_xy
\]
$(F^*g)$ is simply the $\text{Jacobian}^T\cdot \text{Jacobian}$.  

For cylindrical coordinates,
\[
\begin{gathered}
  D_xy = \left[ \begin{matrix} c{\phi } & - rs{\phi} & \\ 
      s{\phi} & rc{\phi} & \\
      & & 1 \end{matrix} \right] \\ 
  \Longrightarrow F^*g = \left[ \begin{matrix} c{\phi } & s{\phi} & \\ 
      -rs{\phi} & rc{\phi} & \\
      & & 1 \end{matrix} \right] \left[ \begin{matrix} c{\phi } & - rs{\phi} & \\ 
      s{\phi} & rc{\phi} & \\
      & & 1 \end{matrix} \right]  = \left[ \begin{matrix} 1 & & \\ & r^2 & \\ & & 1 \end{matrix} \right]
\end{gathered}
\]
\[
\sqrt{ \text{det}(F^*g) } = \sqrt{r^2} =r 
\]
So
\[
\text{div}j = \frac{1}{r} \frac{ \partial (ru^r\rho ) }{ \partial r } +\frac{1}{r} \frac{ \partial (ru^{\varphi } \rho ) }{ \partial \varphi } + \frac{1}{r} \frac{ \partial (ru^z \rho ) }{ \partial z} = \frac{1}{r} \frac{ (\partial ru^r \rho  )}{ \partial r } + \frac{ \partial (u^{\varphi} \rho ) }{ \partial \varphi } + \frac{ \partial (u^z \rho ) }{ \partial z}
\]

\part{Lattice Boltzmann methods (LBM)}

\subsection{Affine Space}

From Pr\'{a}staro (1996) \cite{Pras1996},

\subsubsection{Affine Space discretization}

Consider affine space $(M,\mathbf{M},\alpha)$, and affine frame $(O,\lbrace \mathbf{e}_i \rbrace)$.  $\lbrace \mathbf{e}_i\rbrace$ basis of $\mathbf{M} \xrightarrow{\text{discretize}} \lbrace 0 , 1, \dots L_1 -1\rbrace \times \lbrace 0 ,  \dots L_2-1 \rbrace \times \dots \times \lbrace 0 ,  \dots L_d-1 \rbrace$.

But interestingly, $\mathbf{M} = \lbrace a^i \mathbf{e}_i | a^i \in \mathbb{R} \rbrace \xrightarrow{\text{discretize}} \mathbf{M}$, simply either define $\mathbf{u} \in \mathbf{M}$ as an array of structs or struct of arrays.  




\subsection{Lattice Boltzmann equation, Boltzmann Equation}



\[
\xrightarrow{\text{discretize}} \frac{F_i(\widehat{x},\widehat{t} + \Delta \widehat{t} )- F_i(\widehat{x},\widehat{t} ) }{ \Delta \widehat{t} } + c_i^j \frac{F_i(\widehat{x} + \Delta x^j, \widehat{t} + \Delta \widehat{t} )- F_i(\widehat{x}, \widehat{t} + \Delta \widehat{t} ) }{ \Delta \widehat{x}^j } = \frac{-1}{ \widehat{\tau} \epsilon } (F_i - F_i^{(eq)} )
\]
where $\Delta \widehat{t} = \frac{\Delta t \cdot U}{L}$

Lagrangian behavior obtained by selection of lattice spacing divided by time step to equal lattice velocity:
\begin{equation}
  \frac{ \Delta \widehat{ \mathbf{x}} }{ \Delta \widehat{t}} = \mathbf{c}_i 
\end{equation}
\begin{equation}
  \Longrightarrow \frac{ F_i(\widehat{\mathbf{x}} + \mathbf{c}_i \Delta t, t+ \Delta t ) - F_i(\widehat{\mathbf{x}} , t) }{ \Delta \widehat{t}} = \frac{-1}{ \widehat{\tau} \epsilon }( F_i-F_i^{(eq)} ) \qquad \, (5.14)
\end{equation} (cf. Eq. (5.14) of Wolf-Gladrow \cite{DWol2000})

Choose $\Delta t = t_c$, drop all carats, leads to \emph{(BGK) lattice Boltzmann equation}:
\begin{equation}
  F_i(\mathbf{x}+c_i\Delta t, t+\Delta t) - F_i(\mathbf{x},t) = \frac{-1}{\tau} (F_i - F_i^{(eq)} ) \qquad \, (5.1.5)
  \end{equation}

Pitaevskii, Lifshitz's \textbf{Physical Kinetics} (1981) \cite{PL1981}


Consider 1-dim. sets of discrete velocities $V$, $V\subset \mathbb{Z}$, $Q\in \mathbb{Z}^+$, $Q$ is total number of velocities, i.e. $|V|=Q$.  Consider $Q$ odd, i.e. $Q=2q+1$, $q\in \mathbb{Z}^+_0$.
\[
v_{(i)}\in V, \, v_{(i)} := i \in \mathbb{Z}
\]
By mirror symmetry, $\forall \, v_{(i)} \in V$, $v_{(-i)} \in V$, so $v_{(0)} =0\in V$.  




Succi (2001) \cite{Succ2001}


\section{Lattices, that are ``sufficiently'' Galilean invariant, through non-perturbative algebraic theory}



cf. Appendix A of Karlin and Asinari (2010) \cite{KaAs2010}

From Appendix A of Karlin and Asinari (2010) \cite{KaAs2010}, ``Appendix A: How to Find Closure Relation and Verify Reference Temperature for a given Velocity Set'', as we consider for Maxwell lattices in 1-dim., so that for the underlying affine space $(M,\mathbf{M})$, $M=\mathbb{R}$, $\mathbf{M}=\mathbb{R} \xrightarrow{\text{discretize}} \mathbb{Z}$ in this case.

$Q$ velocities.  Suppose $Q = 2q+1$; $q:= \frac{Q-1}{2}$.

Write
\[
v^Q_{(i)} = a_{Q-2}v_{(i)}^{Q-2} + a_{(Q-4)}v_{(i)}^{Q-4} + \dots + a_1 v_{(i)}
\]
and so in general,
\begin{equation}
  v_{(i)}^Q = \sum_{k=1}^q a_{Q-2k} v_{(i)}^{Q-2k}
\end{equation}
with $q = \frac{Q-2}{2}$, different nonzero values for the velocities.

Use $M_{(n)}^M = b_nT_0^{(n-1)/2} u + O(u^3)$, $n$ odd, $b_n=n!! \equiv 1\cdot 3 \cdot 5 \cdot \dots \cdot n$

Matching condition of linear in $u$ terms results in algebraic equation for reference temperature.
\[
b_QT_0^{(Q-1)/2} - a_{Q-2}b_{Q-2} T_0^{(Q-3)/2} - a_{(Q-4)} b_{Q-4} T_0^{(Q-5)/2} - \dots - a_1 = 0
\]
\begin{equation}
  b_QT_0^{(Q-1)/2} - \sum_{k=1}^{q-1} a_{Q-2k} b_{Q-2k} T_0^{(Q-2k-1)/2} - a_1 = b_Q T_0^{(Q-1)/2} - \sum_{k=1}^q a_{Q-2k} b_{Q-2k} T_0^{(Q-2k-1)/2} 
\end{equation}
since $b_1=1$.

Then determine positive roots for $T_0$.  



Chikatamarla and Karlin (2009) \cite{ChKa2009}

\subsection{Weights}

Consider $f\in C^{\infty}(\mathbb{R}^d)$, the Maxwell-Boltzmann ``distribution'' (or Maxwell-Boltzmann statistics, or Maxwell distribution; there are many names) $f^{MB}\equiv f$.  Consider when $m=1$:
\begin{equation}
\begin{gathered}
  f(v) = \left( \frac{m}{2\pi T_0} \right)^{d/2} \exp{ \left( \frac{ -m(v-u)^2 }{ 2T_0} \right) } \xrightarrow{ m=1} \left( \frac{1}{ 2\pi T_0} \right)^{d/2} \exp{ \left( \frac{ -(v-u)^2}{ 2T_0 } \right)} 
  \end{gathered}
\end{equation}
with $(v-u)^2 = \sum_{i=1}^d (v^i-u^i)^2$


Taking a look at the example of D2Q9 lattice, from pp. 163, Sec. 5.2 BGK lattice Boltzmann model in 2D, Wolf-Gladrow (2000) \cite{DWol2000}, consider the set of discrete velocities $\lbrace c_i \rbrace_{i=0,1,\dots 8}$:





Frapolli, Chikatamarla, and Karlin (2016) \cite{FCK2016} references an analytic solution of a quadratic system of equations, in terms of Lagrange multipliers, for minimizing the entropy $H$, while constraining for the convservation laws (mass, momentum, energy) in their 12, here Ansumali, Karlin, and \"{O}ttinger (2003) \cite{AKO2003}.


\subsection{Reviewing the Entropic Lattice Boltzmann Model for Compressible Flows, with explicit steps/calculations}

From Section II. ``Entropic Lattice Boltzmann Model for Compressible Flows'' of the groundbreaking paper from Frapolli, Chikatamarla, and Karlin (2016) \cite{FCK2016}, the mass, momentum, and (translational) energy conservation laws (relations), in their discrete form, are given by
\begin{equation}
  \lbrace \rho, \rho\mathbf{u}, 2\rho E_{\text{tr}} \rbrace = \sum_{i=1}^Q \lbrace 1, v_i, v_i^2 \rbrace f_i^{\text{eq}}(\rho, \mathbf{u}, T)
  \end{equation}
where $Q = $ total number of discrete velocities (in the model you're considering; this is the $Q$ in $DdQq\equiv dDqQ$).  I will number the discrete velocities from $0$ to $1,\dots Q-1$.  

Indeed, this is the discretized version of the usual conservation laws that may be more familiar in its ``continuous'', integral form:

\[
\begin{aligned}
  \rho & = \int f_i^{\text{eq}}(\rho,\mathbf{u},T) d\mathbf{v} \\ 
  \rho \mathbf{u} & = \int \mathbf{v} f_i^{\text{eq}}(\rho,\mathbf{u},T) d\mathbf{v} \\
  2\rho E_{\text{tr}} = \int v_i^2 f_i^{\text{eq}}(\rho,\mathbf{u},T) d\mathbf{v}
  \end{aligned} \xrightarrow{\text{discretize}}   \lbrace \rho, \rho\mathbf{u}, 2\rho E_{\text{tr}} \rbrace = \sum_{i=1}^Q \lbrace 1, v_i, v_i^2 \rbrace f_i^{\text{eq}}(\rho, \mathbf{u}, T)
\]
keeping in mind that $f_i^{\text{eq}}$ is also a function (dependent upon) $\mathbf{v}$, through a Maxwell-Boltzmann distribution form.  Also, recall the definition of ``translational energy'' for a gas and how it's defined:
\begin{equation}
  2\rho E_{\text{tr}} := 2C_V^{\text{tr}} \rho T + \rho u^2 \qquad \, (\text{translational energy})
\end{equation}
where
\begin{equation}
  C_V^{\text{tr}} = \frac{d}{2} = \text{ specific heat of ideal gas at constant volume }
\end{equation}

Now the discretized form of $H$ is
\begin{equation}
  H=H(f) =\sum_{i=0}^{Q-1} f_i \ln{ \left( \frac{f_i}{W_i} \right)}
  \end{equation}
which is clear from its ``continuous'', integral form (of which is its definition, how it's defined, cf. Sec. 1.3. $H$-theorem of Succi (2001) \cite{Succ2001}):
\begin{equation}
  H=H(f) = -\int f \ln{ \frac{f}{W} } d\mathbf{v}d\mathbf{x} \xrightarrow{ \text{discretize} } H(f\equiv f_0,f_1\dots f_{Q-1}) =\sum_{i=0}^{Q-1} f_i \ln{ \left( \frac{f_i}{W_i} \right)}
  \end{equation}
Considering $f_0,f_1,\dots f_{Q-1}$, the weights $W_i = W_i(T)$, their explicit forms, $\forall \, i = 0,1,\dots Q-1$, are already known, from the algebraic procedure for lattices given above.

So $W_i=W_i(T)$ are calculated explicitly (with numerical values) already at this point.

Now again, recall the conservation laws:
\begin{equation}
\begin{aligned}
  \rho & = \sum_{i=0}^{Q-1} f_i^{\text{eq}}(\rho, \mathbf{u},T) \\ 
  \rho \mathbf{u} & = \sum_{i=0}^{Q-1} v_if_i^{\text{eq}}(\rho,\mathbf{u},T) \\ 
  2\rho E_{\text{tr}} = \sum_{i=0}^{Q-1} v_i^2 F_i^{\text{eq}}(\rho, \mathbf{u},T)
    \end{aligned}
  \end{equation}




\end{multicols*}



\begin{thebibliography}{9}

\bibitem{LLandauELifshitz1987}
L. D. Landau, E.M. Lifshitz, \textbf{Fluid Mechanics}, Second Edition: Volume 6 (Course of Theoretical Physics S) Butterworth-Heinemann, 1987, ISBN-13: 978-0750627672

\bibitem{SAHG1998}
Rolf H. Sabersky, Allen J. Acosta, Edward G. Hauptmann, E.M. Gates.  \textbf{Fluid Flow: A First Course in Fluid Mechanics} (4th Edition).  Prentice Hall. (August 22, 1998).  ISBN-13: 978-0135763728

\bibitem{TFrankel2004}
T. Frankel,
\textbf{The Geometry of Physics}, 
Cambridge University Press, 
Second Edition,
2004.

\bibitem{SMorita2001}
Shigeyuki Morita, \textbf{Geometry of Differential Forms} (Translations of Mathematical Monographs, Vol. 201)  2001


\bibitem{JJost2011}
J\"{u}rgen Jost. \textbf{Riemannian Geometry and Geometric Analysis (Universitext)}. 6th ed. 2011 Edition.  Springer; 6th ed. 2011 edition (August 9, 2011).  ISBN-13: 978-3642212970


\bibitem{OCalinDChang2005}
Ovidiu Calin, Der-Chen Chang. \textbf{Geometric Mechanics on Riemannian Manifolds: Applications to Partial Differential Equations} (Applied and Numerical Harmonic Analysis).  Birkh\"{a}user. 2005. ISBN-13: 978-0817643546







\bibitem{AChorinJMarsden2000}
Alexandre J. Chorin, Jerrold E. Marsden. \textbf{A Mathematical Introduction to Fluid Mechanics} (Texts in Applied Mathematics), Springer; 3rd edition, 2000. ISBN-13: 978-0387979182


\bibitem{HBhatiaGNorgardVPascucciPBremer2013}
Harsh Bhatia, Gregory Norgard, Valerio Pascucci, Peer-Timo Bremer. ``The Helmholtz-Hodge Decomposition—A Survey,'' IEEE Transactions on Visualization and Computer Graphics, Vol. 19,  2013


\bibitem{JLee2012}
John Lee, \textbf{Introduction to Smooth Manifolds} (Graduate Texts in Mathematics, Vol. 218), 2nd edition, Springer,  2012, ISBN-13: 978-1441999818


\bibitem{DHolmTSchmahCStoica2009}
Darryl D. Holm, Tanya Schmah, Cristina Stoica, \textbf{Geometric Mechanics and Symmetry: From Finite to Infinite Dimensions} (Oxford Texts in Applied and Engineering Mathematics) 2009,  ISBN-13: 978-0199212903  ISBN-10: 0199212902  

\bibitem{TKambe2009}
Tsutomu Kambe, \textbf{Geometrical Theory of Dynamical Systems and Fluid Flows}, Advanced Series in Nonlinear Dynamics: Volume 23, 2009, \url{http://www.worldscientific.com/worldscibooks/10.1142/7418} ISBN: 978-981-4282-24-6 (hardcover)

\bibitem{AK1999}
Vladimir I. Arnold, Boris A. Khesin.  \textbf{Topological Methods in Hydrodynamics}. (Applied Mathematical Sciences).  Springer (August 5, 1999).  ISBN-13: 978-0387949475

\bibitem{Pras1996}
Agostino Pr\'{a}staro.  \textbf{Geometry of PDEs and Mechanics}.  World Scientific Publishing Co.  1996.  QC125.2.P73 1996  530.1'55353--dc20.  

\bibitem{DWol2000}
Dieter A. Wolf-Gladrow.  \textbf{Lattice-Gas Cellular Automata and Lattice Boltzmann Models: An Introduction} (Lecture Notes in Mathematics) 2000 Edition.  ISBN-13: 978-3540669739


\bibitem{PL1981}
L. P. Pitaevskii, E.M. Lifshitz. \textbf{Physical Kinetics}: Volume 10 (Course of Theoretical Physics S) 1st Edition.  Butterworth-Heinemann; 1 edition (January 15, 1981).  ISBN-13: 978-0750626354

\bibitem{Succ2001}
 Sauro Succi.  \textbf{The Lattice Boltzmann Equation for Fluid Dynamics and Beyond (Numerical Mathematics and Scientific Computation)} 1st Edition.  Clarendon Press; 1 edition (August 30, 2001).  ISBN-13: 978-0198503989



\bibitem{KaAs2010}
I. Karlin and P. Asinari, \emph{Factorization symmetry in the lattice Boltzmann method}.  \textbf{Physica A} 389, 1530 (2010).  \url{http://staff.polito.it/pietro.asinari/publications/preprint_Asinari_PA_2010a.pdf}.  

See also

Ilya Karlin, Shyam Chikatamarla, Pietro Asinari. \emph{Factorization symmetry in lattice Boltzmann simulations}.  2009.  [arXiv:0911.5529v1](https://arxiv.org/abs/0911.5529v1) [cond-mat.stat-mech] 

\bibitem{ChKa2009}
Shyam S. Chikatamarla and Iliya V. Karlin.  \emph{Lattices for the lattice Boltzmann method}.  \textbf{Physical Review E 79, 046701 (2009)}  
  
\bibitem{FCK2016}
  Nicol\`{o} Frapolli, S. S. Chikatamarla, and Iliya V. Karlin.  \emph{Entropic lattice Boltzmann model for gas dynamics: Theory, boundary conditions, and implementation}.  \textbf{Physical Review E 93}, 063302 (2016).

\bibitem{AKO2003}
Santosh	Ansumali.  Iliya V. Karlin, and H. C. \"{O}ttinger.  \emph{Minimal entropic kinetic models for hydrodynamics}.  \textbf{Europhys. Lett., 63} (6), pp. 798–804 (2003) \\
\url{https://www.researchgate.net/publication/1835231_Minimal_entropic_kinetic_models_for_hydrodynamics}   \\
\url{http://iopscience.iop.org/article/10.1209/epl/i2003-00496-6/meta} \\
\verb|EPL03_1.pdf|










\end{thebibliography}

\end{document}
