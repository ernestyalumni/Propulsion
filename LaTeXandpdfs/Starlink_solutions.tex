% file: Starlink_solutions.tex
%
\documentclass[10pt]{amsart}
\pdfoutput=1
\usepackage{mathtools,amssymb,caption}

\usepackage{graphicx}
\usepackage{hyperref}
\usepackage[utf8]{inputenc}
\usepackage{listings}
\usepackage[table]{xcolor}
\usepackage{pdfpages}
\usepackage{tikz}
\usetikzlibrary{matrix,arrows,backgrounds}

\usepackage{breqn} % for dmath

%\usepackage{cancel} % for Feynman slash notation

\hypersetup{colorlinks=true,citecolor=[rgb]{0,0.4,0}}

%\oddsidemargin=15pt
%\evensidemargin=5pt
%\hoffset-45pt
%\voffset-55pt
%\topmargin=-4pt
%\headsep=5pt
%\textwidth=1120pt
%\textheight=595pt
%\paperwidth=1200pt
%\paperheight=700pt
%\footskip=40pt

\newtheorem{theorem}{Theorem}
\newtheorem{corollary}{Corollary}
%\newtheorem*{main}{Main Theorem}
\newtheorem{lemma}{Lemma}
\newtheorem{proposition}{Proposition}

\newtheorem{definition}{Definition}
\newtheorem{remark}{Remark}

\newenvironment{claim}[1]{\par\noindent\underline{Claim:}\space#1}{}
\newenvironment{claimproof}[1]{\par\noindent\underline{Proof:}\space#1}{\hfill $\blacksquare$}

%This defines a new command \questionhead which takes one argument and
%prints out Question #. with some space.
\newcommand{\questionhead}[1]
  {\bigskip\bigskip
   \noindent{\small\bf Question #1.}
   \bigskip}

\newcommand{\problemhead}[1]
  {
   \noindent{\small\bf Problem #1.}
   }

\newcommand{\exercisehead}[1]
  { \smallskip
   \noindent{\small\bf Exercise #1.}
  }

\newcommand{\solutionhead}[1]
  {
   \noindent{\small\bf Solution #1.}
   }


  \title[SpaceX Starlink solutions]{SpaceX Starlink solutions}

\author{Ernest Yeung \href{mailto:ernestyalumni@gmail.com}{ernestyalumni@gmail.com} and Amir Khan \href{mailto:amirhasandex@gmail.com}{amirhasandex@gmail.com}}
\date{10 Dec 2019}
\keywords{satellites, satellite networks, network topology, topological graph theory, graph theory, algebraic topology}

\begin{document}

\definecolor{darkgreen}{rgb}{0,0.4,0}
\lstset{language=Python,
 frame=bottomline,
 basicstyle=\scriptsize,
 identifierstyle=\color{blue},
 keywordstyle=\bfseries,
 commentstyle=\color{darkgreen},
 stringstyle=\color{red},
 }
%\lstlistoflistings

\maketitle

\tableofcontents

\begin{abstract}
This is an outline of possible solutions and novel approaches towards SpaceX's Starlink, a network of satellites.
\end{abstract}

\section{Introduction}

A casual conversation with Anthony Rose (SpaceX) about the challenges facing SpaceX's Starlink prompted further private discussions amongst the two authors about possible solutions and novel approaches.

\part{Outline of Solutions}

\section{Watchdog Timer} 

Let $i = 0 , 1, \dots N_{\text{WD}} - 1$, where $N_{\text{WD}} = $ total number of Starlink satellites with a Watchdog (WD) timer.

Let $t_{0,i} \equiv t_{0i}$ be the time each Watchdog Timer $i$ gets initialized. This is when the internal watchdog timer begins counting.

Suppose the time duration for a WD timer to "expire" or "timeout" (i.e. once $t_{\text{WD}}$ time elapses, the WD rests to either indicate something went wrong, or on purpose) is chosen to be same $\forall \, i = 0 ,1, \dots N_{\text{WD}} -1$.

either / or 

Because it's not safety critical, but mission assurance.

Let $T_{\text{WD}}$

\section{Network Map}

"Ping loops"

\section{Orbital Parameters}




\begin{thebibliography}{9}

\bibitem{STWD100}
\textbf{Watchdog timer circuit}. Datasheet - production data. STMicroelectronics



\end{thebibliography}

\end{document}
